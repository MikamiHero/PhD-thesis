\chapter{Introduction}
\section{Options}

	In mathematical finance, an option is a contract between two parties (known as the \emph{holder} and the \emph{writer}) that gives the holder the right, but not the obligation, to buy/sell an underlying asset from/to the writer at a mutually agreed price (known as the \emph{exercise} or \emph{strike} price) on or before a specified future date (known as the \emph{expiry} date). On or before the expiry date, the holder may ``exercise'' the option. The right to buy is called a \emph{call} option whereas the right to sell is called a \emph{put} option. Furthermore, a \emph{European} option can only be exercised at expiry whereas an \emph{American} option can be exercised before or on the expiry date. As elementary examples, one can take options on foreign currencies, commodities, or common stock on a firm. An option is known as a financial derivative because it derives its intrinsic value from another entity -- in this case, the underlying asset. This underlying asset's value is often modelled by a stochastic differential equation (SDE) to add a layer of uncertainty. 
	
	A well-known result for determining the option value is known as the Black-Scholes equation \cite{Black1973}. The Black-Scholes formula is in fact a partial differential equation (PDE) that is also paired with a terminal condition governed by a payoff function. The payoff function merely states what the option's value will be at the expiry date. Typically, this is normally a piecewise linear function of the strike price and the underlying asset (at least in terms of a call and put). One of the most classical methods to solve the Black-Scholes PDE system is to invoke a transformation of variables (further details of this in addition to~\cite{Black1973} can be found in~\cite{Wilmott1995}) that effectively eliminates a few terms and reduces the PDE system to the archetypal heat equation commonly seen in thermodynamics~\cite{cannon1984one}. The motivation for this step is because the heat equation has been studied endlessly in literature and the solution to the PDE system is ascertainable~\cite{Churchill1938}. The original details of~\cite{Black1973} can also be complemented with~\cite{Wilmott1995} for further details and discussion about this derivation. The crux of option valuation is to determine a fair price for the holder to pay the writer to enter into their specified contract. This price is commonly known as the \emph{option premium} (or simply, the premium) that actually signifies the value of the option at time-zero (i.e., starting today).
	
For European options, the closed-form analytical solutions are known, but their American option counterparts pose a far greater challenge when attempting to reconcile an exact solution. Due to the flexibility of being able to exercise an American option up to and including the expiry date, there may come a situation where it is in the holder's best interest to exercise this option early once the underlying asset reaches a critical value known as the \emph{optimal exercise price}. The collection of all of these optimal exercise prices for the entire duration of the option's lifetime is denoted the \emph{optimal exercise boundary}. If the holder were to possess these optimal exercise prices, they would have knowledge of when to exercise the contract. However, the optimal exercise boundary is unknown \emph{a priori} and this is what colours it differently to the European option pricing valuation problem. Both the option value and the related optimal exercise boundary are unknown and this consequently introduces complications that require more sophisticated mathematical techniques to solve.

\textcolor{blue}{Financially, American option contracts are more ideal to trade because one can exercise the option at any time up until and including the expiry date}. Perhaps the earliest mathematical formulation for the American option pricing problem was proposed by McKean~\cite{McKean1965} where it was proved that there is an equivalence between the optimal stopping problem describing the framework of the American option valuation problem and a free boundary problem. McKean was then able to derive a homogeneous PDE on a restricted domain, but this was solved using an incomplete Fourier transform and an analytical result was obtained. Geske~\cite{Geske1984} then employed a discrete time approach using a financial derivative called compound options (this will be discussed later in subsection 1.2). The basic idea Geske employed was to first discretize the time domain then price an American put option as the discounted expected value of all future cash flows. The justification for these cash flows is because the put option can be exercised at any discrete interval in time for the duration of the option's lifespan. Another perspective to this is decomposing the American option into a finite quantity of European style options. It is then illustrated that this construction is valid due to it being a solution of the Black-Scholes PDE subject to the free boundary condition imposed by the presence of the optimal exercise boundary.

Although this clever investment strategy of emulating an American put option using cash flows is financially instructive, it does not provide any insight into the properties of the optimal exercise boundary. It was not until the work of Kim~\cite{Kim1990} that not only provided a continuous time solution to the work in~\cite{Geske1984}, but also contained a representation of the American call option as a sum of the European call option plus an additional component that exemplifies the gains one would make from potential early exercise. This term is known as the \emph{early exercise premium}. Moreover, this early exercise premium is expressed in terms of an integral which encapsulates the optimal exercise boundary. If paired with the appropriate boundary conditions with the PDE, one arrives at an integral equation (depending on which type of option is chosen) for the optimal exercise boundary which can then be solved using numerical procedures that involve both a root-finding scheme like Newton-Raphson and numerical integration like a standard quadrature scheme (e.g., Trapezoidal). The aforementioned boundary conditions for the option to determine this integral equation for the free boundary are also known as the \emph{smooth pasting conditions} or matching conditions, and it is meant to mathematically ensure the optimality of the free boundary. The consequent intuitive outcomes by Kim later become coined ``Kim's equations'' or ``Kim's integral equations''. Solving the integral equation related to the optimal free boundary is ideal because if that is completed, then the early exercise premium is also computable and thus, one would be able to obtain the time-zero American option value. As a response to this, several authors have devoted time to investigate and devise algorithms that would solve this integral equation for the exercise premium (in particular, see~\cite{Huang1996, Press1992, Sevcovic2001} just to list a few examples). 

An altered reformulation of the American pricing problem was then later outlined by Jamshidian~\cite{Jamshidian1992}. The analysis resulted in an inhomogeneous PDE in lieu of the homogeneous PDE. The rationale behind this was to extend the restricted domain to an unrestricted domain. Similar to McKean~\cite{McKean1965}, Jamshidian also implemented a Fourier transform methodology to solve this new transformed PDE system and ultimately deduced an integral equation system that identically emulates Kim's integral equations in~\cite{Kim1990}.

In regards to the numerical valuation of American options, the literature is quite rich. We will highlight a few of the more prominent articles and references that have gained popularity for either their simplicity and/or novelty. Cox \emph{et al.}~\cite{Cox1979} developed one of the earliest numerical implementations to price both European and American options called the binomial method. The method involved creating a lattice that is able to track the evolution of the asset's price in a discrete time manner. Each node of the lattice represents a possible price for the underlying asset at a given point in the time domain. The path the underlying asset draws is determined by multiplicative factors that are calcuated using the volatility of the underlying asset. Under the risk-neutral assumption, the option price can be expressed in terms of a discounted expectation of the payoff at expiry. With this, the binomial method starts at the terminal time and invokes the payoff for each possible asset price node. It then marches backwards to inevitably produce the time-zero option price. It is one of the most straightforward algorithms to understand and incorporate into practice.

Monte Carlo simulations have also been another source of numerical promise in relation to American options. Their versatility is in their ability to accommodate for multiple sources of uncertainty in the model. The first venture to incorporate Monte Carlo for the purposes of pricing American-type contingents was by Tilley~\cite{Tilley1993}, who used a dynamic programming approach to establish a scheme that solves for the American option pricing problem backwards in time.

Another class of methods to price American-type derivatives are analytical approximations that account for both the derivative value itself and any embedded features (e.g., American options and the early exercise premium). The most notable technique was proposed by Barone-Adesi and Whaley~\cite{Barone1987} where they developed a quadratic approximation method that replaces the early exercise premium with a term that is quadratic in the underlying asset (with all attached quantities to be either known or able to be determined using known values). Undoubtedly, this would prove beneficial in terms of minimising computational effort and their simulation results prove that the accuracy is also comparable to the finite difference method (FDM) and the compound option approach of Geske. 
	
\section{Compound options}
	A \emph{compound option} is an option on an option. That is, the underlying product is not an asset but another option whose underlying is an asset. For standard vanilla European options, there are four cases: \emph{call-on-a-call, call-on-a-put, put-on-a-call, and put-on-a-put}.

Geske~\cite{Geske1979} developed the seminal theory for compound option models. The motivation was to be able to accurately price a firm's common stock (also known as ordinary share) -- a security that grants the holder corporate equity ownership. Black and Scholes~\cite{Black1973} argued that common stock can be interpreted as an option on the firm. \textcolor{blue}{This is because when a firm defaults}, the holders of the common stock maintain the right but not necessarily the obligation to sell the entire firm to the bondholders (who possess claims to the firm's future cash flows as a result of its financial liabilities) for a strike equal to the face value of the bond \cite{Kwok2008}. Thus an option on a portion of the common stock can be viewed as a compound option since the underlying received upon exercising the option is technically another option on the value of the firm. Consequently, Geske~\cite{Geske1979} derived analytical formulas for compound options when the firm value follows a geometric Brownian motion. This work was then extended to price compound options on various types of bonds (e.g., risky coupon bonds~\cite{Geske1977}, retractable and extendible bonds~\cite{Brennan1977, Ananthanarayanan1980, Longstaff1990}).

A distinct link between compound options and the pricing of American options was introduced by Roll~\cite{Roll1977}. The technique consisted of constructing a portfolio of three European call options: two standard European call options and one European compound call option operating on one of the standard call options. By constructing this replicating portfolio, it is possible to mimic the behaviour of an unprotected American call whose underlying is an asset yielding one known dividend. Roll~\cite{Roll1977} was successfully able to devise an analytical valuation formula for this portfolio by implementing the results from~\cite{Geske1979}. Geske~\cite{Geske1979b} simplified the results in~\cite{Roll1977} and provided a solution which could be readily accommodated for multiple dividends. Whaley~\cite{Whaley1982} provided some corrections to the models presented in~\cite{Roll1977} and~\cite{Geske1979} by noting a misspecification in one of the replicating portfolio terms. This extension is also known as the Roll-Geske-Whaley method for pricing American options.

For non-compound options, a recent extension was provided by Bos and  Wandermark~\cite{Bos2002} that splits the multiple dividends into two categories: ``near'' (dividend payments about to occur) and ``far'' (dividend payments close to expiry). The rationale used was to subtract the ``near'' dividends from the underlying's value and add ``far'' dividends to the strike price. This was an attempt to reconcile methods by Hull~\cite[pp. 298]{Hull1989} and Musiela and Rutkowsky~\cite[pp. 53--54]{Musiela1997}. The former simply subtracted the total value of all dividends in the option's lifetime from the current underlying asset's price. Once this adjustment is made, the option price can be determined. The latter method accumulates the total value of all dividends paid during the lifetime of the option and appends this as a scaling factor to the value of the underlying asset at expiry. From there, one works backwards in time to calculate the option's value. The latter method also assumed the dynamics of the underlying asset to be governed by the Cox-Ross-Rubinstein model~\cite{Cox1979}. Veiga and Wystup~\cite{Veiga2009} notably developed a closed form option pricing formula for assets paying discrete dividends. The formula is expressed in terms of the standard European call option plus a truncated series involving the dividend and derivatives of the aforementioned call (which are obtained by using a Taylor series approximation). The cited works~\cite{Roll1977},~\cite{Geske1979},~\cite{Bos2002}, and~\cite{Veiga2009} formulated the stochastic differential equation (SDE) that models the underlying asset to account for the dividend payment(s) as a term which signifies a cash dividend of a fixed size. This total value of the dividend is not contingent on the value of the underlying asset post-dividend payment date. This distinction between a yield and a cash payment is crucial when constructing the mathematical model to price the underlying asset.

Further attention has been dedicated ever since to valuing American options via the compounds options approach (see \cite{Geske1984},~\cite{Omberg1987},~\cite{Bunch1992},~\cite{Breen1991},~\cite{Ho1997}, \cite{Chang2001}, and \cite{Zhylyevskyy2010}). These cited works assumed that the underlying asset followed a standard diffusion process. Consequently, some analysis has also been conducted in the circumstance where the underlying follows a jump-diffusion process (see~\cite{Gukhal2003} and \cite{Li2005}).


\section{Jump-diffusion models}
	It was verified by Merton \cite{Merton1973b} that one of the fundamental assumptions of the Black-Scholes model is that the asset price follows a continuous-time, diffusion process with a continuous sample path. This prompted Merton in \cite{Merton1976} to consider a ``jump'' stochastic process for the asset price that allows for the probability for it to change at large magnitudes irrespective of the time interval between successive observations. The jumps in the asset price can be accommodated by appending an additional source of uncertainty into the asset price dynamics that models the discontinuity. Moreover, subsequent empirical studies (e.g., Rosenfeld \cite{Rosenfeld1980}, Jarrow \& Rosenfeld~\cite{Jarrow1984}, Ball \& Torous \cite{Ball1985}, and Brown \& Dybvig \cite{Brown1986}) \textcolor{blue}{asserted that the asset price process is best modelled by a stochastic process with a discontinuous sample path}. This phenomenon suggests that the asset price dynamics follow a  \emph{jump-diffusion model}.

Merton~\cite{Merton1976} derived a partial integro-differential equation (or PIDE) to represent a modified Black-Scholes system that accounts for the inclusion of jumps. A solution was also given, which can be viewed as an explicit European option pricing formula in terms of an infinite series of Black-Scholes prices multiplied by a factor that encapsulates the behaviour of the jump. Essentially, the Merton model adds the Poisson process to the Wiener process that governs the asset price. The result is a continuous-time, stochastic process with stationary increments independent of one another, known as a L\'{e}vy process \cite{Platen2010}.

The importance of developing such a system extends beyond attempting to capture the options market's behaviour at any given point. The need lies within being able to deliver fundamental explanations to why certain phenomena occur. For example, when one wishes to estimate the implied volatility surfaces to calibrate the standard Black-Scholes option values to actual market quotes, the Black-Scholes model where the underlying asset follows a standard diffusion process assumes the implied volatility surface to be flat. That is, a constant value during the option's lifetime and for varying values of the strike price (options are commonly listed as a function of their strike price). But empirical observations have shown that these implied volatility surfaces are heavily dependent on both the strike price and the expiry date (in particular, refer to Heynen~\cite{Heynen1994}, Dumas \emph{et al.}~\cite{Dumas1998}, Rebonato~\cite{Rebonato1999}, and Cont \& Fonseca~\cite{Cont2001, Cont2002}). As a result, these surfaces actually form either a ``smile'' or ``skew'' depending on the values of the strike and time to expiry. Dupire~\cite{Dupire1994} developed a technique for computing the local implied volatility surfaces and he showed that the standard Black-Scholes model with an asset under diffusion dynamics can embody all the distinguishing features of this ``smile problem''. However, it only gives us a tool needed to ensure we recover the required option values. It does not explain why these smiles and skews occur. A jump-diffusion model, however, is able to encapsulate both a justification for these smiles and skews, their increased occurrences after the 1987 crash (see Andersen and Andreasen~\cite{Andersen2000}), \textcolor{blue}{and how the jumps in the asset price reflect the "jump fear" in market participants~\cite{Cont2004}}.

In terms of option valuation in jump-diffusion models, the literature is quite rich (e.g., see Amin \cite{Armin1993}, Kou \cite{Kou2002}, Kou \& Wang \cite{Kou2004}, Hilliard \& Schwartz \cite{Hilliard2005}, Carr \& Mayo \cite{Carr2007}, Feng \& Linetsky \cite{Feng2008}, Cheang \& Chiarella \cite{Cheang2011}, and Frontczak \cite{Frontczak2013}) with many resourceful texts (e.g., see Rogers~\cite{Rogers1997}, Kijima~\cite{Kijima2002}, Cont \& Tankov~\cite{Cont2004}, and Vercer~\cite{Vecer2011}).

Amin \cite{Armin1993} developed one of the earliest numerical schemes for pricing options in a jump-diffusion framework by adapting the binomial model proposed by Cox \emph{et al}. \cite{Cox1979}. The extension is achieved by allowing multiple movements in the asset price at every discrete time step to simulate the discontinuous jumps, whereas the standard binomial model allows for only one discrete movement in the asset price at every discrete point in time. \textcolor{black}{This discrete approach is then compared numerically against the closed-form solution provided by Merton~\cite{Merton1976}, with the resultant options values having little differences between one another.}

Pham~\cite{Pham1997} was one of the first to consider pricing American derivatives in a jump-diffusion model. Recall from the introductory subsection about options that the American option pricing problem inherently contains another unknown called the free boundary or optimal exercise boundary. Via a probabilistic approach that utilizes a convexity property of the American option value and a maximum principle, Pham was able to translate the American put option valuation problem to a parabolic integro-differential free-boundary problem. The final result was a decomposition of the American put value as the sum of its European value counterpart and early exercise premium similar in form to the expression found in~\cite{Kim1990}.

In contrast to the probabilistic avenue that was employed in~\cite{Pham1997}, Gukhal~\cite{Gukhal2001} presented analytical formulas for American options under jump-diffusion dynamics through a discrete time method that incorporates compound options similar to what was illustrated in~\cite{Geske1979}. These results account for the underlying asset paying continuous proportional dividends. The rationale Gukhal followed was to construct a American call option by an equivalent portfolio. This portfolio comprised of a European call subjected to the same jump-diffusion process plus the present value of expected dividends in the exercise region, then subtracting the present value of total interest paid on the strike price in the exercise region and a term labelled the ``rebalancing cost'' due to the occurrence of jumps from the exercise region transitioning into the continuation region. Then the time domain is discretized into uniform increments and under the assumption that the American option can only be exercised at a finite number of points in time, an induction argument is proposed to derive a general formula for an American call that is exercisable at an arbitrary value of time instants. Then a limit is taken as the time increment approaches zero and an integral expression for the American call with jumps is ascertained. The study places particular emphasis on the clarity of the analytical results and how they aid in characterising the components of value that contribute towards an American option and how the accommodation for jumps impacts the aforementioned sources. Gukhal then proceeds to analyse specifications of the distribution for the jump amplitude including lognormally distributed jumps and bivariate jumps. 

Further empirical investigations by Kou \cite{Kou2002} led to the proposal of a double exponential jump-diffusion model where the jump intensities are double exponentially distributed. The author's empirical studies contradicted the previous assumptions that the underlying asset's jump-diffusion model was lognormal. Specifically, the findings showed that the return distribution of the asset possessed features uncharacteristic of a normal distribution (i.e., higher peak and heavier asymmetric tails than that of a normal distribution), and the ``volatility smile'' observed in the option markets. Despite the normal distribution being a central mechanism in simulating the asset price process, Kou provided in-depth explanations for the aforementioned empirical analysis and introduced an updated model. This model assumed the jumps in the asset price follow a double exponential distribution. Analytical solutions for pricing of European call/put options and path-dependent options in a double exponential jump-diffusion model were derived in \cite{Kou2004} co-authored with Wang. However, limitations of the model were noted by Kou \cite{Kou2002} in regards to hedging difficulties and assumed dependence of the jump increments.

One key drawback in the Gukhal~\cite{Gukhal2001} formulation was the restriction on the type of payoff function allowed due to the compound option methodology. This was addressed by Chiarella and Ziogas~\cite{Chiarella2006} where they applied Fourier transform techniques in a direct manner to solve the PIDE for an American call and its related free boundary for a jump-diffusion model. This approach mimics that of McKean~\cite{McKean1965} and~Jamshidian~\cite{Jamshidian1992}, as was discussed earlier, for American options in a standard diffusion setting. Chiarella and Ziogas provide both an incomplete Fourier transform McKean approach to solve a homogeneous PIDE on a restricted domain, and a standard Fourier transform Jamshidian scheme that solves an equivalent inhomogeneous PIDE on an unrestricted domain. The authors are also able to reconcile the findings from both methods and provide insightful relations to Gukhal's detailed study in~\cite{Gukhal2001}. 
The integral equations derived for an American call in jump-diffusion dynamics and the free boundary turn out to be interdependent. Numerical simulations are also provided where they also developed a novel extension to the quadrature scheme introduced in~\cite{Kallast2003} to accommodate for the presence of jumps in the model.

In terms of other numerical implementations, Andersen and Andreasen~\cite{Andersen2000} proposed a finite difference method (FDM) to solve the PIDE from Merton~\cite{Merton1976}. They first subject the PIDE to a number of logarithmic transformations then apply simple FDM discretizations to all the derivative terms present in this new PIDE. The discretized equation is then rearranged in a way that resembles the $\theta$-scheme (see Section 12.4.3 in~\cite{Cont2004}). However, Andersen and Andreasen choose to incorporate an alternating direction implicit (ADI) scheme to ensure that their consequent system of difference equations remains a tridiagonal matrix, which will ultimately be more computationally efficient.

With a similar regard to the FDM implementation for jump-diffusion characteristics, d'Halluin \emph{et al.}~\cite{dHalluin2004} developed an implicit discretization method for pricing American options in a jump-diffusion model. They instigated a penalty method similar to~\cite{Forsyth2002} to enforce an American option style of constraint on the pricing formula except with the added condition of the underlying asset ascribe to a jump-diffusion process rather than a standard diffusion process. This type of approach was initially thought to lead to ill-conditionally algebraic problems, but it was demonstrated to be false in~\cite{Forsyth2002}. The process involved discretizing the PIDE but imposing conditions that may change the expression for the discretization (i.e., depending on certain parameter values, the choice of discretization can be either forward, backward, or central). Numerical examples are provided for both the American put and American butterfly options under lognormally distributed jumps.

Hilliard and Schwartz \cite{Hilliard2005} introduced a bivariate tree approach for pricing both European and American derivatives with jumps, where one factor represents a discrete-time version of the standard continuous asset price path whilst the second factor models a discrete-time version of the jumps arriving as a Poisson process. Feng and Linetsky \cite{Feng2008} also provided a computational alternative to pricing options with jumps by introducing a high-order time discretisation scheme to solve the PIDE in Merton's article~\cite{Merton1976}. The authors demonstrated that their method provides rapid convergence to the solution in comparison to standard implicit-explicit time discretisation methods, using Kou's model as a comparative example.

Carr and Mayo \cite{Carr2007} also reported a novel numerical implementation for calculating option prices when the asset is subjected to jump-diffusion dynamics. The authors devised a method that involves converting the integral term in the PIDE derived by Merton \cite{Merton1976} to a correlation integral. They stated that in many instances this correlation integral is a solution to an ordinary differential equation (ODE) or PDE. Carr and Mayo also argued that solving these associated ODEs and PDEs substantially reduces computational effort since it effectively bypasses numerical evaluation of the aforementioned integral. They illustrated their concept by examining both Merton's lognormal model and Kou's double exponential model.

Briefly returning to the analytical side, Cheang and Chiarella \cite{Cheang2011} advocated for amendments to be made to Merton's original jump-diffusion model. They argued that the Merton model makes assumptions that lead to the jump-risk~\cite{Gibson2001} being unpriced and force the distribution of the Poisson jumps to remain unchanged under a change of measure. The authors stressed the significance of this since a realistic market which contains assets with jumps is incomplete. Additionally, when the market price of the jump-risk is accounted for, there exist many equivalent martingale measures that ultimately produce different prices for options. Hence, they introduced a Radon-Nikod\'{y}m derivative process which translates the market measure to an equivalent martingale measure (EMM) for option valuation. However, the EMM is non-unique in the presence of jumps; one must choose the parameters in the Radon-Nikod\'{y}m derivative to establish an EMM to price options. Furthermore, Cheang and Chiarella derived a PIDE and thus a general pricing formula which reduces to Merton's solution \cite{Merton1976} as a special case.

Frontczak \cite{Frontczak2013} adopted a method of solving the PIDE seen in \cite{Merton1976} using Mellin transforms. He proceeded to re-derive Merton's solution for a European put option via direct inversion. Frontczak's approach of directly evaluating the inverse Mellin integral (i.e., a complex integral) \textcolor{black}{is where the approach could be improved}. Moreover, this process needs to be repeated for different payoffs, making this procedure computationally expensive and tedious.
	
	
\section{Implied volatility}
		In the Black-Scholes option pricing model, \textcolor{blue}{most of the associated parameters (e.g., the option price, interest rate) are observable}. The only quantity that cannot be observed is the variable $\sigma$: the volatility. Mathematically, it is significant as its role is to emulate a level of uncertainty in the underlying asset. In practice, its importance is further signified because prior knowledge of $\sigma$ would enable a financial practitioner to accurately price other derivatives that also incorporate the same underlying asset they are interested in. 
		For estimating the volatility~$\sigma$ in the standard diffusion model~\eqref{eqn:stock}, there exist two primary methods. The first scheme involves estimating $\sigma$ from previous asset price movements. That is, suppose a model for the behaviour of the asset involving $\sigma$ is known and the asset prices for all times up until the present are accessible. Then $\sigma$ can be fitted to this observed data. This method is dubbed \emph{historical volatility} as $\sigma$ is approximated using data of previous asset prices. The second is to calibrate all the known parameters, then treat the option value as a function of $\sigma$ (the option price is one of the known parameter values) and solve for $\sigma$. This approach determines $\sigma$ implicitly from the Black-Scholes formula using the option price and the observed parameters, and is referred to as \emph{implied volatility}.  Aside from option pricing in a jump-diffusion framework, another aim of this article is to present a novel implied volatility scheme using Mellin transforms. For the scope of this thesis, we will be limiting the discussions of implied volatility to options where the \emph{vega} (i.e., a measure of sensitivity of options prices to changes in volatility~\cite{Vine2011}) has only one sign. This assumption of a single-signed vega includes European calls and puts.
		
One of the earliest methods for implied volatility estimation was proposed by Latan\'e and Rendleman \cite{Latane76}, where $\sigma$ is computed using a technique called weighted implied standard deviation (WISD). Their idea consisted of obtaining a set of option prices, approximating the implied volatility using the Black-Scholes formula and calculating a WISD using a ``weight" against the Black-Scholes-derived implied volatility. The crux of the method was to reduce any sampling error. Latan\'e and Rendleman concluded the WISD approach was superior in comparison to corresponding historical volatility estimations. Furthermore, the weighting scheme selected provided more weight to options at-the-money and possessing a longer time to expiry.

Cox and Rubinstein \cite{Cox1985} further analysed the weighting scheme proposed by Latan\'e and Rendleman and stressed the importance of employing data from at-the-money options. Their justification was because at-the-money options are the most actively and frequently traded options, thus the implied volatility obtained using at-the-money option values would yield a credible estimation as the data used closely simulates actual trading conditions.

As data from at-the-money options were becoming increasingly appealing to incorporate in implied volatility estimation, Brenner and Subrahmanyam \cite{Brenner1988} introduced a simplified formula for calculating $\sigma$. Their article focused on reducing the complexity of the Black-Scholes pricing formula by assuming the option was at-the-money and close to expiry. These assumptions, coupled with using an asymptotic approximation for the cumulative distribution function (CDF) for a standard normal, resulted in an approximate option valuation formula where $\sigma$ could be evaluated explicitly as a time-constant value. This process allowed one to forego the need to use an iterative procedure to calculate the implied volatility (e.g., the Newton-Raphson method), which was a common practice at the time. The article highlighted that for options close to at-the-money, the value of the option is comparatively proportional to the value of $\sigma$. Furthermore, Brenner and Subrahmanyam stated that their approximation formula may also be implemented as a good initial guess for numerical algorithms like the Newton-Raphson method, since the starting seed is essential for improving the likelihood and speed of convergence \cite{Manaster1982}. The result of Feinstein \cite{Feinstein1988} is nearly identical to Brenner and Subrahmanyan; however, it was developed independently. Curtis and Carriker \cite{Curtis1988} also introduced a closed-form solution for implied volatility estimation for at-the-money options. It can be shown that under certain circumstances, the result by Brenner and Subrahmanyam is a special case of Curtis and Carriker's formula for $\sigma$ (see ``Final remarks'' in \cite{Chargoy2006}).

Despite the resemblance conveyed by at-the-money implied volatility calculations to true trading circumstances, the aforementioned estimations were ill-suited for evaluating implied volatility for option moneyness that is not at-the-money. Studies have been conducted to develop approximations that account for times when the underlying asset price differs from the exercise price (i.e., in-the-money or out-of-the-money options). A notable result was published by Corrado and Miller~\cite{Corrado1996}, where their approximation for $\sigma$ reduces to the Brenner-Subrahmanyam formula for options at-the-money. Their motivation was primarily to improve the accuracy range of implied volatility estimations to a wider scope of option moneyness not necessarily at-the-money. The derivation presented by Corrado and Miller illustrates similarities to that of Brenner and Subrahamyam's approach due to both articles incorporating an asymptotic expansion of the CDF for a standard normal random variable as a gateway to producing simplified approximations. The numerics generated by the authors' result exhibited good agreement with the actual implied volatility via the Black-Scholes formula for options close to and at-the-money. In addition, their numerical output also demonstrated and confirmed that the use of the Brenner-Subrahmanyam result was only accurate for at-the-money options.

Chance \cite{Chance1996} developed an implied volatility approximation that extended the result by Brenner and Subrahmanyam. The author's motivation mimicked that of Corrado and Miller as they derived an expression for $\sigma$ to accommodate for the strike price bias. Chance's formula involved assuming all parameters are known for an at-the-money option, then first deriving an initial guess for $\sigma$ using the Brenner-Subrahmanyan formula (i.e., implied volatility for an at-the-money option). He then demonstrated that the value of an option not at-the-money is simply an at-the-money option perturbed by a value $\Delta v$, which could be the result of differences in strike price and $\sigma$ values. The perturbation $\Delta v$ is then obtained by second-order Taylor expansions resulting in an equation that is quadratic in $\Delta \sigma$. Upon computing $\Delta \sigma$ via the quadratic formula, the final $\sigma$ value for an option not at-the-money is the addition of both $\sigma$ at-the-money plus $\Delta \sigma$. Chance numerically verified the result and illustrated its effectiveness for options near at-the-money (no more than 20 percent in- or out-of-the-money) and options far from expiry. The significance of this was also asserted as long-term options were becoming increasingly popular in practice; however, the author also noted the accuracy decay when the option is closer to expiry. Furthermore, the model requires extra information including an at-the-money option value and its associated Greeks (specifically, vega and the partial derivative with respect to the strike price).

Bharadia \emph{et al.} \cite{Bharadia1996} also reported a result that claimed to be a highly simplified volatility estimation formula, where the primary advantages of the approximation are its simplicity in form and the fact that it does not require the option to be at-the-money.

Amidst these optimistic results, Chambers and Nawalkha \cite{Chambers2001} comparatively examined the implied volatility estimation formulae of Bharadia \emph{et al.}, Corrado and Miller, and Chance. Chambers and Nawalkha praised the result from Bharadia \emph{et al.} for being very condense in form, but also pointed out the inaccuracy (possessing the highest weighted approximation error amongst the three aforementioned estimates). Chambers and Nawalkha commended the Corrado-Miller formula in that an at-the-money option value was not a prerequisite, yet highlighted that the limitation was the square root term (which could be negative). Furthermore, whether the formula would produce a complex solution is unknown \emph{a priori}; however, the likelihood is minimised substantially for reasonable parameter values. Chambers and Nawalkha accommodated for the possibility of a negative argument for the square root term by setting the term to be zero if the case occurred. It was commented that Corrado and Miller's model is extremely accurate for options near at-the-money, but substantial errors are prevalent for options very far from at-the-money. Corrado and Miller's formula for $\sigma$ possessed the second highest weighted approximation error.

Special attention was devoted to Chance's estimate in \cite{Chambers2001} as it produced the lowest weighted approximation error amongst the three models. The assessment of Chance's approximation gave positive mention of accuracy and ease of understanding/implementation. Similar to Corrado and Miller, Chance's formula yields the highest accuracy for near at-the-money options, but deteriorates for options significantly far from at-the-money.
This consequently provided the mathematical structure for Chambers and Nawalkha's result, developing a simplified extension to Chance's formula that dramatically improved the accuracy.

Chambers and Nawalkha attempted to improve the accuracy of Chance's implied volatility model for options relatively far from at-the-money (where all three formulas suffered in accuracy). Recall that Chance employed a second-order Taylor series expansion in two variables as there was justification for both the strike price and volatility to contribute to the change in option prices. Chambers and Nawalkha adopted a similar approach by performing a second-order Taylor Series expansion around $\Delta v$, but only with respect to $\sigma$. The result was a much simpler quadratic equation in $\Delta \sigma$ and similar to Chance's formula, required an initial guess for $\sigma$ at-the-money (which is also computed via Brenner and Subrahmanyam's approximation). Chambers and Nawalkha asserted that the effect of strike price differences can be encapsulated in the Brenner-Subrahmanyam formula for $\sigma$, thus only requiring the partial derivative with respect to volatility in the Taylor series expansion. The weighted approximation error is ultimately the lowest in comparison to the three models by Chance, Corrado and Miller, and Bharadia \emph{et al.} Despite this, Chambers and Nawalkha's method shares the same detriment to Chance's formula in that an option value at-the-money is required to estimate a starting $\sigma$ value.

From all the schemes presented above, the common hindrance is either the need for additional data (e.g., at-the-money option value) or the deterioration of accuracy for options very far from at-the-money. Li \cite{SLi2005} attempted to rectify the need for extra information and improved reliability for options deep in- or out-of-the-money. By incorporating a substitution of variables and Taylor series expansion, Li derived two separate formulas for $\sigma$ depending on whether the option was at-the-money or not. The numerical results provided in Li's paper demonstrate greater accuracy for $\sigma$ for both at-the-money and not-at-the-money scenarios. Interestingly, Li's approximation for $\sigma$ not at-the-money reduces to the Brenner-Subrahmanyan formula under special conditions. Several other approaches have also been developed over recent years (see Park \emph{et al.} \cite{Park2011}, Choi \emph{et al.} \cite{Choi2011}, Zhang \& Man \cite{Zhang2014}, Chen \& Xu \cite{Chen2014}).
\textcolor{black}{
For the majority of the aforementioned cited works, the primary motivation was to develop an analytical approximation for the implied volatility that possessed benefits over their predecessors. Although many of the methods were derived from seemingly ad hoc methodologies, the validity of these analytical results is still valuable as it provides us with a means to evaluate and analyse the sensitivity of the implied volatility to the other financial parameters. If one took a standard iterative (e.g., Newton-Raphson) numerical approach to compute the implied volatility, it may be difficult to gauge how the behaviour of this obtained $\sigma$ value varies with the other parameters.}
		
\section{The Mellin transform}
		Perhaps the most fundamental mathematical technique that essentially connects all the content in this thesis together is the Mellin transform. Named after its creator, the Finnish mathematician Hjalmar Mellin, this integral transform can be related to the two-sided Laplace transform through an exponential change of variables~\cite{Bracewell2000} and can also be linked to the Fourier transform~\cite{Cohen2008}. Historically, it has used consistently in many applications in number theory due to its ability to exploit the analytic properties of the Riemann zeta function, which aided in one proof of the prime number theorem~\cite[pp. 51--54]{Titchmarsh1986}. Additionally, its usage has seen versatility in other fields of mathematics including the asymptotics of Gamma-related functions seen in complex function theory~\cite{Mellin1904}, analysis of Dirichlet series as seen in number theory~\cite{Friedberg2006, Montgomery2007}, and in statistics for studying the distribution of products and quotients of independent random variables~\cite{Epstein1948}.
		In terms of mathematical finance, one of the earliest sightings of the Mellin transform was by Cruz-B\'aez and Gonz\'alez-Rodr\'iguez~\cite{Cruz2002} where they incorporated semigroup theory to show the existence and uniqueness of European options when the associated parameters (i.e., risk-free interest rate, constant dividend yield, and volatility) were independent of time. In particular, they made use of the inverse Mellin transform, but their solution still remained as a complex-valued integral without further simplification. Panini and Srivastav~\cite{Panini2003} studied how the Mellin transform could aid in pricing \emph{basket options}. Taking a slight detour, a basket option is an option that acts upon a collection (or basket) of stocks (sometimes referred to as a \emph{rainbow option} because the different ``colours'' can represent different underlying assets). An example of this can be to create a basket option that trades two or more different foreign currencies. Basket options are particularly favourable as a hedging mechanism due to the total volatility of a basket option being lower than the individual volatilities of each of the underlying assets~\cite{Jiang2005}. Similar to~\cite{Cruz2002}, the authors of~\cite{Panini2003} also assumed the coefficients of the Black-Scholes system to be time-independent functions. The next most notable work was by J\'odar \emph{et al.}~\cite{Jodar2005} where the authors directly applied the Mellin transform to the Black-Scholes PDE system and performed the necessary analysis. They also conducted all their evaluations without the need to instantiate a change of variables, which is a common step when dealing with the Black-Scholes PDE as it reduces the problem to the standard heat equation form. This in turn makes it easier to solve, but having to keep track of these variable changes may prove to be quite tedious. However like~\cite{Cruz2002} and~\cite{Panini2003}, the results in~\cite{Jodar2005} also assumed time-independent forms for the coefficients associated to the Black-Scholes framework. Similarly, their final solution was also expressed in terms of a complex integral as a consequence of instigating the inverse Mellin transform to invert from the Mellin space to the original variable space.
		The analysis to include time-dependent coefficients in the Black-Scholes model was not accounted for until the comprehensive work of Rodrigo and Mamon~\cite{Rodrigo2006}. Their findings included the rigorous existence-uniqueness proof of European options in the Black-Scholes framework with time-dependent coefficients. Furthermore, the exposition also asserts that the results are under quite general European contingent claims -- that is, for a general payoff function that is not confined to a European put or call option. This is also the research that introduced the Black-Scholes kernel, which will be used quite extensively in this thesis. To validate their general results, Rodrigo and Mamon examined the payoffs for a European put and call to show that the analysis is consistent with the well known identities present in the literature. Rodrigo~\cite{Rodrigo2013} then extended the previous seminal work by analysing vanilla American options. Through the Mellin transform techniques, Rodrigo was able to derive the American put and call option pricing formula in the integral form that was originally accredited to Kim~\cite{Kim1990}. Moreover, the primary focus of the study was to derive an approximate ODE for the optimal free boundary. This ODE can be solved in isolation without the need to pair it with its corresponding option pricing formula that is apparent in the American option pricing problem. Rodrigo derived this result by borrowing a concept from fluid mechanics known as the K\'arm\'an-Pohlausen technique that deals with the thickness of a boundary layer~\cite[pp. 421--423]{Zwillinger1998}. As was previously mentioned in the section regarding jump-diffusion models, Frontczak \cite{Frontczak2013} incorporated the Mellin transform to solve the PIDE system seen in~\cite{Merton1976} which was achieved via a brute-force inversion. Although the final result is in terms of real functions, the inversion process (which involves complex-valued integrals) needs to be repeated if one changed the payoff function for the option, thus making this scheme somewhat cumbersome in this regard.
		The Mellin transform and its affiliated techniques/identities will be seen repeatedly in this thesis. Although it can act in a Laplace/Fourier framework, we choose to utilize the Mellin transform in its own space as it possesses some elegant properties when dealing with functions multiplied by its derivatives. Specifically, there are terms in the Black-Scholes PDE and Merton's PIDE that have a form that meshes well with some of the Mellin transform results.
		
		
		
		
\section{Outline of the thesis}
		The thesis will be structured as follows. All the necessary preliminary knowledge will be contained in chapter 2. This will cover the introductory content for the standard European and American options as well as options on underlying assets that are subjected to jump-diffusion dynamics, and the Mellin transform. These will form the \textcolor{blue}{basis} for all the work to come. 
		
		In chapter 3 will contain the alternative results to pricing options and evaluating implied volatility in jump-diffusion models. Our approach implements the Mellin transform similar to \cite{Frontczak2013} to derive the necessary results. The structure will be as follows. We provide the main results (i.e., the derivations) and demonstrate specific cases to the main results, associated verifications, and other pertinent analogous relations. The highlight of this chapter will be deriving an exact analytical expression for the jump-diffusion component, which to our knowledge has not been achieved up until now. Concluding remarks will also be given to summarise and discuss these alternative formulas.
		 
		The implied volatility content commences in chapter 4, where our main focus is determining an expression for the implied volatility under the assumption that jumps are present in the underlying asset price process. We begin by deriving a Dupire-like PIDE. This is then followed by deriving the implied volatility formula required. Once again, this is under jump-diffusion dynamics. Numerics will be analysed and investigated to assess the potential application of the aforementioned results. Finally, we will present a discussion of the findings followed by a conclusion with tentative future directions.
		
		As an extension to the work in chapter 3, we will be investigating the jump-diffusion model for American options in chapter 5. Specifically, the main focus here will be to develop a method and associated algorithm to solve for the optimal moving boundary (or exercise boundary) that separates the continuation and stopping/exercise regions for American options. The result is an approximate ordinary differential equation that is similar in form to the one Rodrigo derived in~\cite{Rodrigo2013}. The accuracy of this new system will be compared against a the finite-difference method for American options in jump-diffusion dynamics. This is then succeeded by a discussion of the findings and concluding remarks.
		
		In chapter 6, we propose a method for computing European compound options with general payoffs assuming the underlying asset follows a geometric Brownian motion. We also show that for standard vanilla European options, we only need to generate two cases; the remaining two can be ascertained via a \emph{pseudo-put-call parity} technique. We can also obtain a combined result that resembles the standard put-call parity identity for compound options, and this stems from using the pseudo-put-call parity approach to derive the four standard vanilla compound option cases. We then extend to pricing European compound options when the underlying pays one discrete dividend. \textcolor{blue}{Although} analytical formulas for European compound options with continuous dividend yields for the underlying have been derived before in the literature, to our knowledge this has not been done for discrete paying dividends either as a cash payment or proportional yield of the underlying asset. The approach we adopt in this work incorporates the Black-Scholes kernel and the integral identities associated with it to give an exact pricing formula for compound options on a underlying that has a continuous dividend yield. With a slight modification, we also show that this approach is valid when the underlying pays one discrete dividend. We will model the dynamics of the discrete dividend payment as a yield of the underlying asset.

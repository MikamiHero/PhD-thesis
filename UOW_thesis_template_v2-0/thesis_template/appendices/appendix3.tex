\chapter{MATLAB code used throughout the thesis}
\chaptermark{MATLAB code for the thesis}

\section{Code to output Black-Scholes' standard diffusion and Merton's jump-diffusion option profiles}
\begin{lstlisting}
function optionComp

%% Financial parameters %%
K = 100;
r = 0.05;
q = 0.00;
T = 0.5;
sigma = 0.3;
S0 = 20:0.5:500;

%% Jump parameters %%
lambda = 0.5;
mu = -0.90;
delta = 0.45;
iter = 100;
alpha = exp(mu + 0.5*delta^2) - 1;
lambdaF = lambda*(1+alpha);

%% Useful functions %%
N = @(x) 0.5 + 0.5*erf(x/sqrt(2));
z1 = @(x,t,u) (log(x) + (r-q+0.5*sigma^2).*(u-t)) ./ (sigma*sqrt(u-t));
z2 = @(x,t,u) (log(x) + (r-q-0.5*sigma^2).*(u-t)) ./ (sigma*sqrt(u-t));
call = @(x,t) -K.*exp(-r.*(T-t)).*N(z2(x./K,t,T)) ...
        + x.*exp(-q.*(T-t)).*N(z1(x./K,t,T));

%% Computing the jump-diffusion European call option price %%
Cj = zeros(length(S0),1);
for i = 1:length(S0)
    sum = 0;
    for n = 0:iter
        rF = r - lambda*alpha + (n*log(1+alpha))/T;
        sigmaF = sqrt(sigma^2 + (n*delta^2)/T);
        d1 = (log(S0(i)./K) + (rF - q + 0.5*sigmaF^2)*T)/(sigmaF*sqrt(T));
        d2 = d1 - sigmaF*sqrt(T);
        JF = ((lambdaF*T)^n/factorial(n))*exp(-lambdaF*T);
        CBSM = S0(i).*exp(-q*T)*N(d1) - K*exp(-rF*T)*N(d2);
        sum = sum + JF*CBSM;
    end
    Cj(i) = sum;
end
C = call(S0(:),0);

%% Plotting the results %%
set(gca,'FontSize',18,'fontWeight','bold')
set(findall(gcf,'type','text'),'FontSize',18,'fontWeight','bold')

figure (1);
plot(S0,C,'r',S0,Cj,'b--');
legend('Black-Scholes','Lognormal jumps',4);
axis([0 500 0 max(Cj)]);
title('Black-Scholes versus Merton''s model with lognormal jumps');
xlabel('Asset price (S_0)');
ylabel('Call option value (V_0(S_0,0))');
grid on;

figure (2);
plot(S0,Cj-C);
title('Difference between Merton`s model and Black-Scholes');
xlabel('Asset price (S_0)');
ylabel('Call option difference (v_M - v)');
grid on;
end
\end{lstlisting}

\section{Code to compute the implied volatility in a jump-diffusion model}
\subsection{Implied volatility using theoretical data}

\begin{lstlisting}
function impvolJumpTheo

%% Input parameters
S = 15;
K = linspace(10^(-6),8*S,500);
r = 0.05;
q = 0.03;
sigma = 0.30;
T = 0.25;
t = 0.0;

%% Jump parameters
lambda = 0.1;
mu = -0.9;
delta = 0.45;

%% Number of iterations
iter = 30;

%% Parameters for log-normal jump
alpha = exp(mu + 0.5*delta^2) - 1;

%% Modified EU B-S option price
N = @(x) 0.5 + 0.5*erf(x/sqrt(2));
lambdaF = lambda*(1+alpha);
E = @(x) exp(mu.*x + 0.5.*(delta.*x).^2);

%% Computing options prices with jumps for varying strike prices
V = zeros(1,length(K));
for j = 1:length(K)
    sum = 0;
    for n = 0:iter
        rF = r - lambda*alpha + (n*log(1+alpha))/(T-t);
        sigmaF = sqrt(sigma^2 + (n*delta^2)/(T-t));
        d1 = (log(S./K(j)) + (rF - q + 0.5*sigmaF^2)*(T-t))/...
            (sigmaF*sqrt(T-t));
        d2 = d1 - sigmaF*sqrt(T-t);
        JF = ((lambdaF*(T-t))^n/factorial(n))*exp(-lambdaF*(T-t));
        CBSM = S*exp(-q*(T-t))*N(d1) - K(j)*exp(-rF*(T-t))*N(d2);
        sum = sum + JF*CBSM;
    end
    V(j) = sum;
end

%% Computing the Mellin transform
ksi = 1.0:0.25:5.0;
sigmaNum = zeros(1,length(ksi));
phi = @(x) (S.^(x+1))./(x.*(x+1));

for i = 1:length(ksi)
    xi = ksi(i);
    Y = K(:).^(xi-1).*V(:);
    euroint = trapz(K,Y);
    A = (log(euroint./phi(xi)))./(xi*(xi+1)*(T-t));
    B = (r - q - alpha*lambda)/(xi+1);
    C = ((q + alpha*lambda) - lambda*(E(xi+1)-1))/(xi*(xi+1));
    
    sigmaNum(i) = sqrt(2*(A - B + C));
end

%% Plotting
set(gca,'FontSize',12)
plot(ksi,sigmaNum);
axis([1 5.0 0 0.31]);
grid on;
xlabel('\xi value');
ylabel('Implied volatility');
title('Implied volatility estimation for options in a jump-diffusion model');

end

\end{lstlisting}

\subsection{Implied volatility estimation using pseudo-market data}
\begin{lstlisting}
    function impvolJumpPseu

%% Input parameters
S = 15;
K = linspace(5,25,20);
r = 0.05;
q = 0.03;
sigma = 0.30;
T = 0.25;
t = 0.0;

%% Jump parameters
lambda = 0.1;
mu = -0.9;
delta = 0.45;

%% Number of iterations
iter = 30;

%% Parameters for log-normal jump
alpha = exp(mu + 0.5*delta^2) - 1;

%% Modified EU B-S option price
N = @(x) 0.5 + 0.5*erf(x/sqrt(2));
lambdaF = lambda*(1+alpha);
E = @(x) exp(mu.*x + 0.5.*(delta.*x).^2);


V2 = zeros(1,length(K));
for j = 1:length(K)
    sum = 0;
    for n = 0:iter
        rF = r - lambda*alpha + (n*log(1+alpha))/(T-t);
        sigmaF = sqrt(sigma^2 + (n*delta^2)/(T-t));
        d1 = (log(S./K(j)) + (rF - q + 0.5*sigmaF^2)*(T-t))/...
            (sigmaF*sqrt(T-t));
        d2 = d1 - sigmaF*sqrt(T-t);
        JF = ((lambdaF*(T-t))^n/factorial(n))*exp(-lambdaF*(T-t));
        CBSM = S*exp(-q*(T-t))*N(d1) - K(j)*exp(-rF*(T-t))*N(d2);
        sum = sum + JF*CBSM;
    end
    V2(j) = sum;
end

%% Extrapolation
% Make sure jump terms are 0
% Head
beta = (S.*exp(-q*(T-t)) - V2(1))./K(1);
head = @(x) S.*exp(-q*(T-t))-beta.*x;

KHead = linspace(10^(-6),K(1),10);
VHead = head(KHead);

% Tail
KL = 8*S;
D = log((V2(end-1)*(KL-K(end)))./(V2(end)*(KL-K(end-1)))) ./ ...
    (log(K(end)) - log(K(end-1))); 
gamma2 = (V2(end).*K(end).^D)./(K(end)-KL);
gamma1 = -KL.*gamma2;
tail = @(x) gamma1./(x.^D) + gamma2./(x.^(D-1));

KTail = linspace(K(end),KL,10);
VTail = tail(KTail);

% Recombining everything
V3 = [VHead(1:end-1) V2 VTail(2:end)];
K3 = [KHead(1:end-1) K KTail(2:end)];

%figure;
%plot(K3,V3);

%% Computing the Mellin transform

ksi = 1.0:0.25:5.0;
sigmaNum = zeros(1,length(ksi));

phi = @(x) (S.^(x+1))./(x.*(x+1));

for i = 1:length(ksi)
    xi = ksi(i);
    func = pchip(K3,V3);
    euroint = quadgk(@(x) (x.^(xi-1)).*ppval(func,x),K3(1),K3(end));
    
    A = (log(euroint./phi(xi)))./(xi*(xi+1)*(T-t));
    B = (r - q - alpha*lambda)/(xi+1);
    C = ((q + alpha*lambda) - lambda*(E(xi+1)-1))/(xi*(xi+1));
    
    sigmaNum(i) = sqrt(2*(A - B + C));
end

figure;
set(gca,'FontSize',12)
plot(ksi,sigmaNum);
axis([1 5.0 0 0.31]);
grid on;
xlabel('\xi value');
ylabel('Implied volatility');
title('Implied volatility estimation for options in a jump-diffusion model');

end

\end{lstlisting}

\subsection{Newton's method}
\begin{lstlisting}
    function impvoln(K)

%% Input parameters
S0 = 15;
r = 0.05;
q = 0.03;
sigma = 0.30;
T = 0.25;

%% Setting up EU call price function parameters
N = @(z) 0.5 + 0.5.*erf(z./sqrt(2));
d1 = @(sig) (log(S0./K)+(r-q+0.5*sig.^2).*T)./(sig.*sqrt(T));
d2 = @(sig) (log(S0./K)+(r-q-0.5*sig.^2).*T)./(sig.*sqrt(T));

%% Computing the observed option price to use in the imp. vol. method
C_obs = S0.*exp(-q.*T).*N(d1(sigma)) - K.*exp(-r.*T).*N(d2(sigma));

%% Initial seed for Newton's method (using critical value of sigma)
sigma0 = sqrt(abs((2./T)*(log(S0./K) + (r-q)*T)));

%% Error tolerance
epsilon = 10^(-6);

%% Initial error (> epsilon)
error = 1.0;

%% First value for Newton's method
sNow = sigma0;

%% Loop for Newton's method
while (error > epsilon)
    fprintf('We"re in\n');
    D1 = d1(sNow);
    D2 = d2(sNow);
    f = S0.*exp(-q.*T).*N(D1) - K.*exp(-r.*T).*N(D2) - C_obs;
    vega = S0.*exp(-q.*T).*sqrt(T)*(1./sqrt(2*pi))*exp(-D1.^2 / 2);
    
    sNext = sNow - (f ./ vega);
    error = abs(sNext - sNow);
    sNow = sNext;
end

%% Final value from loop = implied volatility
sigmaImp = sNow;
disp([K C_obs sigmaImp]);

end
\end{lstlisting}

\subsection{Brenner-Subrahmanyam}
\begin{lstlisting}
    function impvolBS

%% Parameters and functions
S0 = 15;
K = 5;
r = 0.05;
q = 0.03;
sigma = 0.3;
T = 0.25;
t = 0.0;

N = @(x) 0.5 + 0.5*erf(x/sqrt(2));

d1 = (log(S0./K) + (r - q + 0.5*sigma^2)*(T-t))/(sigma*sqrt(T-t));
d2 = (log(S0./K) + (r - q - 0.5*sigma^2)*(T-t))/(sigma*sqrt(T-t));
V = S0.*exp(-q*(T-t))*N(d1)-K*exp(-r*(T-t))*N(d2);

sigmaBrenner = sqrt(2*pi/T).*(V./S0);

end
\end{lstlisting}

\subsection{Corrado-Miller}
\begin{lstlisting}
    function impvolCM
%% Parameters and functions
S0 = 15;
K = 5;
r = 0.05;
q = 0.03;
sigma = 0.3;
T = 0.25;
t = 0.0;

N = @(x) 0.5 + 0.5*erf(x/sqrt(2));
d1 = (log(S0./K) + (r - q + 0.5*sigma^2)*(T-t))/(sigma*sqrt(T-t));
d2 = (log(S0./K) + (r - q - 0.5*sigma^2)*(T-t))/(sigma*sqrt(T-t));
V = S0.*exp(-q*(T-t))*N(d1)-K*exp(-r*(T-t))*N(d2);

%% Corrado and Miller
sigmaCorrado = (1./(S0+K)).*sqrt(2*pi/T).* ...
    (V - (S0-K)./2 + sqrt((V - (S0-K)/2).^2 - ((S0-K).^2)/pi));
end
\end{lstlisting}

\section{Code to aid in pricing an American put option subjected to jump-diffusion dynamics}
\subsection{FDM to solve Merton's PIDE}
\begin{lstlisting}
function [S,v] = fdmPIDEAmerPut

%% Numerical Parameters
I = 101;                                                                  % # of nodes in asset price domain
N = 501;                                                                   % # of nodes in time domain

%% Financial parameters
T = 0.5;                                                                        % Time to expiry
K = 5;                                                                       % Strike price
r = 0.05;                                                                      % Interest rate
q = 0.0;                                                                      % Continuous dividend yield
sigma = 0.3;                                                                 % Volatility
S0 = 10^(-3);
Smax = 4*K;
S = linspace(S0,Smax,I);
t = linspace(0,T,N);
dT = abs(t(2)-t(1));

%% Jump parameters (lognormal)
mu = 0.90;
delta = 0.45;
lambda = 0.1;

%% Some simplifications to make life easier
sig2 = sigma.^2;
del2 = delta.^2;

%% Computing kappa = E[Y-1]
kappa = exp(mu+0.5.*del2) - 1;

%% Lognormal PDF
f = @(y) 1./(y.*sqrt(2*pi*del2)).*exp(-0.5.*((log(y)-mu).^2./(del2)));

%% Initializing matrix
v = zeros(N,I);

%% Setting the boundary conditions (S --> 0  and S --> Smax)
v(:,1) = K;
v(:,I) = 0;

%% Setting the terminal (final) condition
v(N,:) = max(K-S,0);

for n=N:-1:2
    for i=2:(I-1)
        sum = 0;
        for j=1:I
            sum = sum + v(n,j).*f(j./i);
        end
        u1 = v(n,i) + 0.5.*dT.*sig2.*(i.^2).*(v(n,i-1)-2.*v(n,i)+v(n,i+1)) ...
            + 0.5.*dT.*i.*(r-q-kappa*lambda).*(v(n,i+1)-v(n,i-1)) ...
            - dT.*(r+lambda).*v(n,i) + dT.*lambda/i.*sum;
        u2 = max(K-S(i),0);
        v(n-1,i) = max(u1,u2);
    end
end
end
\end{lstlisting}

\subsection{Code to solve for the American put option in jump-diffusion dynamics using $S^*$ from Appendix C.3.2.}
\begin{lstlisting}
    function [Va] = amerPutJ2(Sf)
%% Financial parameters
T = 0.5;                                                                     % Time to expiry
t = 0.0;                                                                     % Time-zero 
K = 5;                                                                     % Strike price
%S0 = [K 2*K 3*K 4*K 5*K 6*K 7*K 8*K]/4;                                      % Value of asset at time zero
%S = [90 92 94 96 98 100 106 112 118];
S = linspace(10^-3,4*K,101);
%S = 100;
r = 0.05;                                                                    % Interest rate
q = 0.00;                                                                    % Continuous dividend yield
sigma = 0.3;                                                                 % Volatility

%% Jump parameters (lognormal)
mu = 0.90;
delta = 0.45;
lambda = 0.1;

%% Some simplifications to make life easier
sig2 = sigma.^2;
del2 = delta.^2;
alpha = lambda.*exp(-mu-sig2./2);
kappa = exp(mu+0.5.*del2)-1;
lambdaF = lambda*(1+alpha);

%% Numerical parameters
I = length(Sf);
time = linspace(t,T,I);
iter = 30;

%% Useful functions
N = @(x) normcdf(x);
Np = @(x) 1./(sqrt(2*pi)).*exp(-x.^2/2);
z2n = @(x,t,u,n) (log(x) + n.*mu + (r-q-kappa*lambda-0.5*sig2).*(u-t)) ... 
 ./(sqrt(n*del2 + sig2.*(u-t+eps)));
z1n = @(x,t,u,n) (log(x) + n.*mu + n.*del2 + (r-q-kappa*lambda+0.5*sig2) ...
    .*(u-t))./(sqrt(n*del2 + sig2.*(u-t+eps)));
C = @(x,p,u,n,t) sqrt((n+1).*del2 + sig2.*(u-t))./ ...
    (delta.*sqrt(n.*del2+sig2.*(u-t+eps)))...
    .*(log(x) - n.*mu - (r-q-kappa.*lambda-sig2/2).*(u-t) ...
    + (sqrt(n*del2 + sig2.*(u-t))./((n+1).*del2 + sig2.*(u-t))) ...
    .*(log(1./(x.*p)) + (n+1).*mu + (r-q-kappa*lambda-sig2/2).*(u-t)));


%% Standard EU put without jumps as an estimate we'll use
Pe = @(x,t,u) K.*exp(-r.*(u-t)).*N(-z2n(x./K,t,u,0)) - ...
     x.*exp(-q.*(u-t)).*N(-z1n(x./K,t,u,0));

%% Computing the American put option with jumps
P = zeros(1,length(S));
Va = zeros(1,length(S));
for m = 1:length(S)
    for n = 0:iter
        rF = r - lambda*alpha + (n*log(1+alpha))/(T-t);
        sigmaF = sqrt(sigma^2 + (n*delta^2)/(T-t));
        d1 = (log(S(m)./K) + (rF - q + 0.5*sigmaF^2)*(T-t))/(sigmaF*sqrt(T-t));
        d2 = d1 - sigmaF*sqrt(T-t);
        JF = ((lambdaF*(T-t))^n/factorial(n))*exp(-lambdaF*(T-t));
        PBSM = K*exp(-rF*(T-t))*N(-d2) - S(m).*exp(-q*(T-t))*N(-d1);
        P(m) = P(m) + JF*PBSM;
    end
end

%% Computing the early exercise premium. 
%% I could have it under one loop, but this is easier to see
for m = 1:length(S)
    rTerm = @(u) 0;
    qTerm = @(u) 0;
    jTerm = 0;
    for n = 0:iter
        if n==0
            Y1 = @(u) exp(-(r+lambda).*(u-t)).*N(-z2n(S(m)./Sf,t,u,n));
            Y2 = @(u) exp(-(q+lambda+kappa*lambda).*(u-t)).*N(-z1n(S(m)./ ...
                Sf,t,u,n));
        else
            Y1 = @(u) ((lambda.*(u-t)).^n)./(factorial(n))...
                .*exp(-(r+lambda).*(u-t)).*N(-z2n(S(m)./Sf,t,u,n));
            Y2 = @(u) ((lambda.*(u-t)).^n)./(factorial(n))...
                .*exp(-(q+lambda+kappa*lambda).*(u-t)).*N(-z1n(S(m)./ ...
                Sf,t,u,n));
        end
        rTerm = @(u) rTerm(u) + Y1(u);
        qTerm = @(u) qTerm(u) + Y2(u);
    
        F = zeros(1,length(Sf));
        for j=1:length(Sf)
            
            B = ((lambda.^(n+1)).*(time(j)-t).^n)./(factorial(n).*sqrt((n+1)* ...
                del2 + sig2.*(time(j)-t+eps)));
            val = @(p) (1./p).*(Pe(Sf(j).*p,t,time(j))-(K-Sf(j).*p)).* ...
                Np(z2n(S(m)./(Sf(j).*p),t,time(j),n+1))...
                .*N(C(Sf(j)./S(m),p,time(j),n,t));

            F(j) = B.*exp(-(r+lambda).*(time(j)-t)).*quadgk(val,1,Inf);
        end
        intY3 = trapz(time,F);
        jTerm = jTerm + intY3;
    end
    Y1 = rTerm(time);
    Y2 = qTerm(time);
    Va(m) = P(m) - r.*K.*trapz(time,Y1) + q.*S(m).*trapz(time,Y2) + jTerm;
end
end
\end{lstlisting}
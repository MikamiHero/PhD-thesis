\documentclass{article}
\usepackage{amsmath}
\usepackage{amssymb}
\usepackage{color}
\usepackage{mathrsfs}
\usepackage[margin=0.9in]{geometry}
\usepackage{url}
\usepackage{enumerate}
\usepackage{dsfont}
\usepackage{hyperref}

\begin{document}
RE: Response to external examination reports for Tianyu Raymond Li's PhD
\\
\\
School of Mathematics and Applied Statistics \\
University of Wollongong \\
Wollongong, New South Wales \\
Australia 2522
\\
\\
5 Dec 2017
\\
\\
Dear A/Prof Adam Rennie,
\\
\\
I would like to thank you once again for taking the time to handle the administrative processes involved with submitting my thesis for external examination and together with my supervisors (Dr Marianito Rodrigo and A/Prof Joanna Goard), sifting through the examination reports and constructing a report for me. 
\\
\\
I have attached below a point-by-point response to each of the external examiner's comments and suggestions. The two examiners were Prof Matt Davison of Western University (Canada) and Dr Jeff Dewynne of University of Oxford (UK). I wish to highlight a two points from Matt Davison have not been directly addressed as they were simply statements rather than changes to be made. Those have been omitted.
\\
\\
Kindest regards,
\\
\\
Tianyu Raymond Li
\\
\\
\\
\textcolor{blue}{\textbf{Responses to Prof Matt Davison\\ \\}}
\begin{enumerate}
    \item \textbf{Page 1 first sentence of the last paragraph. I would say we are talking about the Black-Scholes equation here-- it is fairly common usage to reserve the phrase ``Black-Scholes formula'' for expressions like $C(S,t) = SN(d_1) - Ke^{-rt}N(d_2)$.
    \\
    \\}
    The phrase ``Black-Scholes formula'' in this sentence has been changed to read ``Black-Scholes equation''.
     \item \textbf{Page 3 2nd line ``early early"' $\rightarrow$ ``early".
    \\
    \\}
    The typo has been fixed.
     \item \textbf{Page 12 2nd last line ``directly evaluating of the Inverse Mellin...'' $\rightarrow$ ``directly evaluating the inverse Mellin...''.
    \\
    \\}
    The typo has been fixed.
     \item \textbf{Page 20 last line ``Dupre'' $\rightarrow$ ``Dupire''.
    \\
    \\}
    The typo has been fixed.
     \item \textbf{Page 24 2nd last line after eq (2.8) -- I would simplify the sentence ``As aforementioned in the introduction'' to simply read ``As mentioned in the introduction''.
    \\
    \\}
    The sentence now reads ``As mentioned in the introduction''.
      \item \textbf{Page 24 bottom of the page. While I agree that the smooth pasting conditions can arise from a profit maximizing argument, and even how this makes the $V(S^*(t),t) = K - S^*(t)$ half of the smooth pasting condition essential, it is harder to understand why the derivative term of the smooth pasting condition corresponds to maximization. In fact, I prefer to think of the derivative term as arising from the need to have a $\Delta$ of -1 at put exercise to avoid any risk in the hedged porfolio. 
    \\
    \\}
    Additional references have been provided here to reinforce the financial significance of the smooth pasting conditions. A proof has also been cited.
    
      \item \textbf{Page 25 Line 2, superfluous space between ``formulation'' and the period.
    \\
    \\}
    The unwarranted space has been removed.
    
      \item \textbf{Page 31 statement of Lemma 3 ``it follows from that'': follows from what?.
    \\
    \\}
    This is a typo. It has been changed to read ``it follows that''.
    \item
     \textbf{Page 33 statement of Lemma 8 -- don't italicize ``and''. Line following eq (2.43) -- not sure why some words are italicized here and not others.
    \\
    \\}
    The inconsistent italics have been removed.
    \item \textbf{Page 33/34 Lemma 8 ``defined to be the discretized Black-Scholes kernel (pg 33)''. I don't love this nomenclature as the kernel from Lemma 8 (2.44) seems perfectly continuous to me. Can a different name be found here?
    \\
    \\}
    The reason for the word ``discretized`` is to differentiate it from the expression for the Black-Scholes kernel that corresponds to an asset that pays a continuous dividend yield versus a discrete dividend payment. We believe the expression is suitable as it is and has not been altered. 
    
     \item \textbf{Page 51 -10 lines from the end -- I think you mean ``$H$ functions'' not ``$Hh$ functions''? Of course the link between $H$ functions and $G$ functions and the Mellin transforms is very fundamental.
    \\
    \\}
    This is no typo. We were indeed referring to $Hh$ functions that Kou~\cite{Kou2002,Kou2004} implemented to handle the jump-diffusion dynamics following a double exponential distribution.
    
     \item \textbf{Page 55 first line of section 4.1.2 ``are in terms of the strike price'' $\rightarrow$ ``are with respect to the strike price''.
    \\
    \\}
    The phrase has been changed to the recommendation.
    
         \item \textbf{Page 63 I feel there are way too many significant figures here.
    \\
    \\}
    The number of significant figures was a by-product of using \emph{format long} in MATLAB (which is what we used to generate our data). Additionally, these results have already been accepted and published. Thus, we will be keeping the data as is.
    
         \item \textbf{Page 73 last line -- change to: ``is because, upon inversion, this...''.
    \\
    \\}
    The last line has been changed to what was recommended.
    
         \item \textbf{Page 95, line 5 ``was done in [26]" $\rightarrow$ ``was investigated in [26]'' or ``was reported in [26]".
    \\
    \\}
    This has been changed to ``was reported in''.
    
         \item \textbf{Page 77 5th last line ``instigating'' the properties of the Delta function? Perhaps you mean the properties of the Delta function?
    \\
    \\}
    This has been addressed and changed to ``using the properties of the Delta function''.
    
         \item \textbf{Page 100 2nd last line before section 5.4.1 ``renownedly'' $\rightarrow$ ``renowned''.
    \\
    \\}
    This typo has been fixed.\\\\\\

\end{enumerate}

\textcolor{blue}{\textbf{Responses to Dr Jeff Dewynne\\ \\}}
\begin{enumerate}
	\item \textbf{
	Major points}
	\begin{itemize}
		\item \textbf{\underline{Section 1.4} -- Firstly, it is assumed that there is a \emph{unique} implied volatility in this section. That is true (assuming that certain no-arbitrage bounds hold) for European puts and calls, but in general it is not true; for example, it is easy to show that a European digital put does not necessarily have a unique implied volatility. Please make it clear you are talking about European puts and calls in this section (or, more generally, about options where the vega, $\partial V / \partial \sigma$, has only one sign).\\ \\}
		This specificity has been appended to Section 1.4 on pg.~13 of the revised thesis (the end of the 1st paragraph).
		
		\item \textbf{\underline{Section 1.4} -- Secondly, nowadays most people would compute the (Black-Scholes) implied volatility for European puts and calls numerically. This section seems to discuss a large number of special approximations, with analytic formulae, that only work in various special cases. It does not seem to discuss how one might compute the implied volatility in a way that \emph{always} works.
		\\ \\}
		A general all-purpose method for computing implied volatility is mentioned on pg.~14 and pg.~18 of the revised thesis. Specifically, it references the Newton-Raphson scheme that is mentioned in~\cite{Manaster1982} (which was also a recommend reference by Dr Dewynne).
		
		\item \textbf{\underline{Section 1.5} -- If would be a very good idea to state what the Mellin transform is and, just as importantly, what the inverse transform is. It should be made very clear that in general $\xi$ is a \emph{complex} variable, i.e., $\xi \in \mathbb{C}$.
		\\ \\}
		The Mellin transform is explicitly stated in detail in Section 2.3 starting on pg.~30 and ending on pg.~31 of the revised thesis. We chose to not include it in the introduction as this is purely a descriptive section that consists of only a literature review. We purposefully withheld from including any mathematics until the preliminaries in Chapter 2. Additionally, we made the change $\xi \in \mathbb{C}$ as corrected on pg.~30.
		
		\item \textbf{ \underline{Section 2.1} -- Please either prove the smooth pasting conditions,
			$$
				v_\text{a}^\text{put}(S^*(t),t) = K - S^*(t), \quad \frac{\partial v_\text{a}^\text{put}}{\partial x}(S^*(t),t) = -1,
			$$
			or give a reference a proof.\\ \\}
			A proof has been provided via a reference. This addition can be seen on pg.~26 of the revised thesis.
			
		\item \textbf{\underline{Section 3.4} -- In (3.14), how do you know that
			$$
				\int_0^\infty \frac{1}{z}F_1(z)F_{n-1}\left(\frac{x}{z}\right) \, \d z,
			$$
		for $n \geq 2$ exists? \\ \\}
		
		\item \textbf{\underline{Section 3.4} -- I tried (3.14) out for the log-Cauchy distribution $Y = e^X$ where $X$ is a Cauchy distributed variable with probability density function
		$$
			p(x) = \frac{1}{\pi}\frac{1}{1+x^2}.
		$$
		The probability density function for $Y > 0$ is
		$$
			f(y) = \frac{1}{\pi y}\frac{1}{1 + (\log(y))^2}
		$$
		and following (3.14) one eventually arrives at
		$$
			F_n(x) = \frac{n}{\pi} \frac{1}{n^2 + (\log(x))^2}
		$$
		for $n = 1,2,3, \ldots.$ This was about the worst example I could come up with, as the Cauchy distribution has no variance (and, according to some purists, it doesn't even have a mean). In view of this, I am fairly sure you \emph{must} be able to prove that (3.14) works for a fairly general class of pdfs.\\ \\}
		
		Any family of jumps will be compatible with our analysis in \S3 provided $\mathbb{E}[Y]$ (and consequently $\mathbb{E}[Y-1]$) is finite. If this were not true, (2.20), (2.21) -- the general PIDE system for a general European option with jump-diffusion dynamics -- cannot be posed in the first place as $\kappa = \mathbb{E}[Y-1]$ would not be defined. 
	
	Dr Dewynne points out that the aforementioned log-Cauchy distribution does give reasonable results. Since it does not have a defined mean, this type of distribution does not conform to the assumption that the $Y$ has a finite expectation. To make things clearer, we have added the assumptions that $E[Y]$ must be finite and $E[Y^{-\xi}]$ is convergent to the start of Section 3.2 on pg.~40 of the revised thesis.
					\end{itemize}
\end{enumerate}

\begin{thebibliography}{100}
\bibitem{Kou2002} Kou, S. (2002): ``A jump-diffusion model for option pricing'', \emph{Management Science}, Vol. 48, No. 8, pp.~1086--1101.
	\bibitem{Kou2004} Kou, S. and H. Wang (2004): ``Option pricing under a double exponential jump-diffusion model'', \emph{Management Science}, Vol. 50, No. 9, pp. 1178--1192.
	\bibitem{Manaster1982} Manaster, S. and G. Koehler (1982): ``The calculation of implied variances from the Black-Scholes model: A Note'', \emph{The Journal of Finance}, Vol. 37, No. 1, pp. 227--230

\end{thebibliography}
\end{document}
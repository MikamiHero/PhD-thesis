\documentclass{article}
\usepackage{amsmath}
\usepackage{amssymb}
\usepackage{color}
\usepackage{mathrsfs}
\usepackage[margin=0.9in]{geometry}
\usepackage{url}
\usepackage{enumerate}
\usepackage{dsfont}
\usepackage{hyperref}

\begin{document}
RE: Response to external examination reports for Tianyu Raymond Li's PhD
\\
\\
School of Mathematics and Applied Statistics \\
University of Wollongong \\
Wollongong, New South Wales \\
Australia 2522
\\
\\
5 Dec 2017
\\
\\
Dear A/Prof Adam Rennie,
\\
\\
I would like to thank you once again for taking the time to handle the administrative processes involved with submitting my thesis for external examination and together with my supervisors (Dr Marianito Rodrigo and A/Prof Joanna Goard), sifting through the examination reports and constructing a report for me. 
\\
\\
I have attached below a point-by-point response to each of the external examiner's comments and suggestions. The two examiners were Prof Matt Davison of Western University (Canada) and Dr Jeff Dewynne of University of Oxford (UK). Any changes made to the revised thesis are now in \textcolor{blue}{blue}. I wish to highlight two points from Matt Davison have not been addressed as they were simply statements rather than changes to be made. Those have been omitted.
\\
\\
Kindest regards,
\\
\\
Tianyu Raymond Li
\\
\\
\\
\textcolor{blue}{\textbf{Responses to Prof Matt Davison\\ \\}}
\begin{enumerate}
    \item \textbf{Page 1 first sentence of the last paragraph. I would say we are talking about the Black-Scholes equation here-- it is fairly common usage to reserve the phrase ``Black-Scholes formula'' for expressions like $C(S,t) = SN(d_1) - Ke^{-rt}N(d_2)$.
    \\
    \\}
    The phrase ``Black-Scholes formula'' in this sentence has been changed to read ``Black-Scholes equation''.
     \item \textbf{Page 3 2nd line ``early early"' $\rightarrow$ ``early".
    \\
    \\}
    The typo has been fixed.
     \item \textbf{Page 12 2nd last line ``directly evaluating of the Inverse Mellin...'' $\rightarrow$ ``directly evaluating the inverse Mellin...''.
    \\
    \\}
    The typo has been fixed.
     \item \textbf{Page 20 last line ``Dupre'' $\rightarrow$ ``Dupire''.
    \\
    \\}
    The typo has been fixed.
     \item \textbf{Page 24 2nd last line after eq (2.8) -- I would simplify the sentence ``As aforementioned in the introduction'' to simply read ``As mentioned in the introduction''.
    \\
    \\}
    The sentence now reads ``As mentioned in the introduction''.
      \item \textbf{Page 24 bottom of the page. While I agree that the smooth pasting conditions can arise from a profit maximizing argument, and even how this makes the $V(S^*(t),t) = K - S^*(t)$ half of the smooth pasting condition essential, it is harder to understand why the derivative term of the smooth pasting condition corresponds to maximization. In fact, I prefer to think of the derivative term as arising from the need to have a $\Delta$ of -1 at put exercise to avoid any risk in the hedged porfolio. 
    \\
    \\}
    Additional references have been provided here to reinforce the financial significance of the smooth pasting conditions. A proof has also been cited.
    
      \item \textbf{Page 25 Line 2, superfluous space between ``formulation'' and the period.
    \\
    \\}
    The unwarranted space has been removed.
    
      \item \textbf{Page 31 statement of Lemma 3 ``it follows from that'': follows from what?.
    \\
    \\}
    This is a typo. It has been changed to read ``it follows that''.
    \item
     \textbf{Page 33 statement of Lemma 8 -- don't italicize ``and''. Line following eq (2.43) -- not sure why some words are italicized here and not others.
    \\
    \\}
    The inconsistent italics have been removed.
    \item \textbf{Page 33/34 Lemma 8 ``defined to be the discretized Black-Scholes kernel (pg 33)''. I don't love this nomenclature as the kernel from Lemma 8 (2.44) seems perfectly continuous to me. Can a different name be found here?
    \\
    \\}
    The reason for the word ``discretized`` is to differentiate it from the expression for the Black-Scholes kernel that corresponds to an asset that pays a continuous dividend yield versus a discrete dividend payment. We believe the expression is suitable as it is and has not been altered. 
    
     \item \textbf{Page 51 -10 lines from the end -- I think you mean ``$H$ functions'' not ``$Hh$ functions''? Of course the link between $H$ functions and $G$ functions and the Mellin transforms is very fundamental.
    \\
    \\}
    This is no typo. We were indeed referring to $Hh$ functions that Kou~\cite{Kou2002,Kou2004} implemented to handle the jump-diffusion dynamics following a double exponential distribution.
    
     \item \textbf{Page 55 first line of section 4.1.2 ``are in terms of the strike price'' $\rightarrow$ ``are with respect to the strike price''.
    \\
    \\}
    The phrase has been changed to the recommendation.
    
         \item \textbf{Page 63 I feel there are way too many significant figures here.
    \\
    \\}
    The number of significant figures was a by-product of using \emph{format long} in MATLAB (which is what we used to generate our data). Additionally, these results have already been accepted and published. Thus, we will be keeping the data as is.
    
         \item \textbf{Page 73 last line -- change to: ``is because, upon inversion, this...''.
    \\
    \\}
    The last line has been changed to what was recommended.
    
         \item \textbf{Page 95, line 5 ``was done in [26]" $\rightarrow$ ``was investigated in [26]'' or ``was reported in [26]".
    \\
    \\}
    This has been changed to ``was reported in''.
    
         \item \textbf{Page 77 5th last line ``instigating'' the properties of the Delta function? Perhaps you mean the properties of the Delta function?
    \\
    \\}
    This has been addressed and changed to ``using the properties of the Delta function''.
    
         \item \textbf{Page 100 2nd last line before section 5.4.1 ``renownedly'' $\rightarrow$ ``renowned''.
    \\
    \\}
    This typo has been fixed.\\\\\\

\end{enumerate}

\textcolor{blue}{\textbf{Responses to Dr Jeff Dewynne\\ \\}}
\begin{enumerate}
	\item \textbf{
	Major points}
	\begin{itemize}
		\item \textbf{\underline{Section 1.4} -- Firstly, it is assumed that there is a \emph{unique} implied volatility in this section. That is true (assuming that certain no-arbitrage bounds hold) for European puts and calls, but in general it is not true; for example, it is easy to show that a European digital put does not necessarily have a unique implied volatility. Please make it clear you are talking about European puts and calls in this section (or, more generally, about options where the vega, $\partial V / \partial \sigma$, has only one sign).\\ \\}
		This specificity has been appended to Section 1.4 on pg.~13 of the revised thesis (the end of the 1st paragraph).
		
		\item \textbf{\underline{Section 1.4} -- Secondly, nowadays most people would compute the (Black-Scholes) implied volatility for European puts and calls numerically. This section seems to discuss a large number of special approximations, with analytic formulae, that only work in various special cases. It does not seem to discuss how one might compute the implied volatility in a way that \emph{always} works.
		\\ \\}
		A general all-purpose method for computing implied volatility is mentioned on pg.~14 and pg.~18 of the revised thesis. Specifically, it references the Newton-Raphson scheme that is mentioned in~\cite{Manaster1982} (which was also a recommend reference by Dr Dewynne).
		
		\item \textbf{\underline{Section 1.5} -- If would be a very good idea to state what the Mellin transform is and, just as importantly, what the inverse transform is. It should be made very clear that in general $\xi$ is a \emph{complex} variable, i.e., $\xi \in \mathbb{C}$.
		\\ \\}
		The Mellin transform is explicitly stated in detail in Section 2.3 starting on pg.~30 and ending on pg.~31 of the revised thesis. We chose to not include it in the introduction as this is purely a descriptive section that consists of only a literature review. We purposefully withheld from including any mathematics until the preliminaries in Chapter 2. Additionally, we made the change $\xi \in \mathbb{C}$ as corrected on pg.~30.
		
		\item \textbf{ \underline{Section 2.1} -- Please either prove the smooth pasting conditions,
			$$
				v_\text{a}^\text{put}(S^*(t),t) = K - S^*(t), \quad \frac{\partial v_\text{a}^\text{put}}{\partial x}(S^*(t),t) = -1,
			$$
			or give a reference a proof.\\ \\}
			A proof has been provided via a reference. This addition can be seen on pg.~26 of the revised thesis.
			
		\item \textbf{\underline{Section 3.4} -- In (3.14), how do you know that
			$$
				\int_0^\infty \frac{1}{z}F_1(z)F_{n-1}\left(\frac{x}{z}\right) \, \d z,
			$$
		for $n \geq 2$ exists? \\ \\}
		
		\item \textbf{\underline{Section 3.4} -- I tried (3.14) out for the log-Cauchy distribution $Y = e^X$ where $X$ is a Cauchy distributed variable with probability density function
		$$
			p(x) = \frac{1}{\pi}\frac{1}{1+x^2}.
		$$
		The probability density function for $Y > 0$ is
		$$
			f(y) = \frac{1}{\pi y}\frac{1}{1 + (\log(y))^2}
		$$
		and following (3.14) one eventually arrives at
		$$
			F_n(x) = \frac{n}{\pi} \frac{1}{n^2 + (\log(x))^2}
		$$
		for $n = 1,2,3, \ldots.$ This was about the worst example I could come up with, as the Cauchy distribution has no variance (and, according to some purists, it doesn't even have a mean). In view of this, I am fairly sure you \emph{must} be able to prove that (3.14) works for a fairly general class of pdfs.\\ \\}
		
		Any family of jumps will be compatible with our analysis in \S3 provided $\mathbb{E}[Y]$ (and consequently $\mathbb{E}[Y-1]$) is finite. If this were not true, (2.20), (2.21) -- the general PIDE system for a general European option with jump-diffusion dynamics -- cannot be posed in the first place as $\kappa = \mathbb{E}[Y-1]$ would not be defined. 
	
	Dr Dewynne points out that the aforementioned log-Cauchy distribution does give reasonable results. Since it does not have a defined mean, this type of distribution does not conform to the assumption that the $Y$ has a finite expectation. To make things clearer, we have added the assumptions that $E[Y]$ must be finite and $E[Y^{-\xi}]$ is convergent to the start of Section 3.2 on pg.~40 of the revised thesis.
	
	\item\textbf{\underline{Section 3.4 and 3.5} -- As it currently stands, I don't really see the point of these. You haven't proved that all the integrals in (3.14) exist and so how I do I know that if I can find all of the $F_n$ using iteration. \\ \\ }
	
	The examples of double exponentially distributed jumps and gamma distributed jumps are just being presented here for comparison to the results in~\cite{Kou2002, Kou2004} and ~\cite{Frontczak2013}. Please also see the point above where we have explained which family of distributions $F_n$ is valid for (with a proof as well).
	
	\item\textbf{\underline{Section 4.2} -- You start with the assumption that the (Black-Scholes) volatility is a constant regardless of the call's strike. This is in direct conflict with all the empirical studies: implied volatility always depends non-trivially on the strike, hence the expressions ``volatility smile'' and ``volatility smirk'', for example. Therefore, I fail to see why this section is entitled `Implied volatility formula'.
	\\\\ }
	We have removed the erroneous statement that included $\sigma$ to be constant along with the parameters $r$ and $q$.
	
	\item{\textbf{\underline{Section 4.2} -- What this section is really about is the ability to interpolate and extrapolate solutions (with the same constant value of the volatility) in such a way that you can take their Mellin transform and then recover the constant volatility that you used to generate the prices in the first place from the Mellin transforms. All this shows is that you have good schemes for interpolation and extrapolation and your numerical means of computing the Mellin transform works. \\ \\ }}
	This is indeed true and we agree with Dr Dewynne's statements here. This assertion can be seen in Section 4.3 where we describe the extrapolation process to recover the original value of $\sigma$ used to compute the options prices profile with which we took the Mellin transform of.
	
	\item{\textbf{\underline{Section 4.2} -- You  only use \emph{pseudo}-market data (without jumps, so far as I can see) in this and the following sections. In spite of the (unjustified) claim in \S4.3.3, Point 1, you could \emph{not} do this with \emph{real} market data because the (Black-Scholes implied) volatility of European call pries (on the same underlying asset and with the same expiry date) \emph{does} change as you change the strike, breaking one of your underlying assumptions. This section has nothing to do with implied volatility in the usually understood sense of the term. \\ \\}}
	
	 The implied volatility model we have derived in Section 4.2 and in particular, Eq.~(4.8) \emph{does} contain jumps (as symbolised by the $\lambda$ and $\mathbb{E}[Y^{\xi+1}-1]$). We provided two sets of simulations: theoretical data (Tables 4.1 and 4.2) and pseudo-market data (Tables 4.3 and 4.4). In \S4.3.4, we computed Eq.~(4.8) assuming the jumps were drawn from a lognormal distribution. The associated parameters are also provided here. The pseudo-market simulations were completed assuming no jumps to demonstrate the concept of the extrapolating functions (which we also highlighted can be adapted for jumps in \S4.3.2). Thus, we have in fact incorporated jump-diffusion terms in Eq.~(4.8); we simply chose to demonstrate both scenarios of jumps and no jumps.
	 
	 For this comment, it appears that Dr Dewynne's primary concern is the underlying assumption that volatility $\sigma$ is independent of the strike price $K$ which is in direct conflict with the usual assumptions/evidence towards implied volatility being dependent on $K$. However, if assumed $\sigma = \sigma(K)$ (which is a known fact for market data), we would not have been able to take the Mellin transform in the first place if this were the case, so we had to make this assumption. We acknowledge that this is a limitation of our method; we are only able to obtain one implied volatility value as opposed to obtaining a volatility surface. Although other common numerical schemes may prove fruitful in computing a volatility surface like a Newton-Raphson solver, it is not entirely clear how $\sigma$ can change if you change parameters (e.g, $r$ and $q$) other than $K$ and $T$. However, in Eq.~(4.8), we have constructed the model in a way that makes it very easy to see how $\sigma$ would vary if you were to change the aforementioned parameters. 
	 
	 \item{\textbf{\underline{Section 5.1.1} -- Why does (5.3) hold in a jump-diffusion model? I suggest you simply, and explicity, say that you are \emph{assuming} that (5.3) and (5.6) hold for American puts and that (5.19) and (5.22) hold for American calls.\\ \\}}
	 This assumption of the smooth pasting conditions to be satisfied for both American calls and American puts in a jump-diffusion model has been added to pg.~72 and pg.~86 of the revised thesis.
	 
	 \item{\textbf{\underline{Section 5.2 and 5.3} -- Have you looked at perpetual American put and call (assuming constant $r$, $q$, $\sigma$) with jump diffusion? Any analytical formula that arose from perpetual option cases could shed some light on formulae (5.35) and (5.36). I find these formulae slightly disturbing because they seem to imply that a jump occurs right at expiry (otherwise, if no jump occurs at expiry, why should the formulae be any different to the pure diffusion based ones?).\\\\}}
	 
	 Perpetual American puts and calls were not investigated as this was not included in our scope of research. Eq.~(5.35) was simply derived using the technique found in~\cite{Chiarella2006} but adapted for American puts. Eq.~(5.36) is found in~\cite{Chiarella2006}.
					\end{itemize}
					
	\item{\textbf{Minor points}}
		\begin{itemize}
			\item{\textbf{Page 2, final paragraph: ``Financially, American options contracts are more ideal to trade...". I would just say that you can exercise an American option at any time up until and including the expiry date.
			\\\\}
			The rewording has been done on pg.~2 of the revised thesis.}
			
			\item{\textbf{Page 5, penultimate paragraph: ``This is because when a firm decides to default, ...". It is not clear to me that a firm \emph{decides} to default -- either it can meet its obligations or it can't.
			\\\\}
			This has been changed to ``This is because when a firm defaults'' on pg.~5 of the revised thesis.}
			
			\item{\textbf{Page 7, first paragraph in \S1.3: ``...asserted that the asset price is best resembled by a stochastic process..." would read better as ``... asserted that the asset price is best \emph{modelled} by a stochastic process...".
			\\\\}
			This change has been made on pg.~7 of the revised thesis.}
			
			\item{\textbf{Page 8, bottom of the first paragraph: ``and how the jumps in the asset price bear some psychological parallel to the potential tentative demeanour of the market participants [33]''. I have no idea what that phrase means nor what it is you're trying to convey by writing it.
			\\\\}
			This has been fixed to read ``and how the jumps in the asset price reflect the "jump fear" in market participants'' on pg.~8 of the revised thesis.}
			
			\item{\textbf{Page 9, final paragraph: ``These acclaimed results account for the underlying asset paying...''. I would eliminate the word \emph{acclaimed}.
			\\\\}}
			This has been removed.
			
			\item{\textbf{Page 13, start of \S1.4: It is not the case that \emph{all} of the associated parameters are observable. The \emph{volatility} is an associated parameter and it is \emph{not} directly observable; that is the one of the points of using \emph{implied volatility}. (Also, what is the difference between a \emph{variable} and an \emph{intrinsic variable}?)
			\\\\}}
			The word ``intrinsic'' has been removed and the phrase now says ``most of the associated parameters (e.g., the option price, interest rate) are observable'' on pg.~13 of the revised thesis.
			
			\item{\textbf{Page 20, last sentence in the first paragraph of \S1.6: The sentence would read better if \emph{rudiment} were changed to \emph{basis}.
			\\\\}}
			The word \emph{rudiment} has been changed to \emph{basis} on pg.~20 of the revised thesis.
			
			\item{\textbf{Page 21, final paragraph: Why is the technique called a ``\emph{pseudo}-put-call parity technique''? The word \emph{pseudo} implied that it isn't a genuine technique and as far as I can see, it is a perfectly genuine technique. The word \emph{general} or similar, seems to be a better choice than \emph{pseudo}.\\\\}}
			The word \emph{pseudo} was selected to distinguish it between the standard put-call parity technique. It isn't meant to discredit the technique but rather say that for compound options, you can obtain put-call parity-like results but are not commonly defined as such (to the best of our knowledge). Thus, we've chosen to use \emph{pseudo} and will keep it like this for the thesis. 
			
			\item{\textbf{Page 21, replace \emph{Albeit} with \emph{Although}\\\\}}
			The word \emph{albeit} is now \emph{although}.
			
			\item{\textbf{Page 21, Finally, what you are actually end up doing in Chapter 6, \S6.4, is applying a discrete dividend yield, i.e., you assume that
			$$
				D(S_t,t) = q_dS_t\delta(t-t_d).
			$$
			Why not just called it a discrete dividend yield? \\\\}}
			
			We simply used the terminology that was given as the title in~\cite{Wilmott1993}, which has in \S6.2.3 the title of \emph{discrete dividend payments}. The value  $q_d$ itself is referred to as the \emph{yield}. 
			
			\item{\textbf{Page 23 \S2.1: Why is it ``under \emph{any} risk-neutral probability measure''? Surely in this case there is only one risk-neutral measure?\\\\}}
			This has been change to read ``under the risk-neutral probability measure'' on pg.~23 of the revised thesis.
			
			\item{\textbf{Page 22, \S2.1: Do you really need $r(t) > 0$ and $q(t) \geq 0$ in this part of the thesis, where you're just formulating the Black-Scholes equation?\\\\}}
			These are assumptions that we will be using throughout the rest of the thesis, so it makes sense for us to establish them here.
			
			\item{\textbf{Page 22, \S2.1: Why do $r(t)$, $q(t)$ and $\sigma(t)$ need to be continuous functions, rather than piecewise continuous (for example) or even just bounded and integrable?\\\\}}
			They can be piecewise continuous.
			
			\item{\textbf{Page 22, \S2.1: The phrase ``(i.e., $S_t$ \emph{ascribes} to a lognormal...'' would read better as ``(i.e., $S_t$ \emph{has} a lognormal...". Similarly, the phrase ``The value $\sigma(t)$ is often \emph{coined} the diffusion..." would ready better as ``The value $\sigma(t)$ is often \emph{called} the diffusion...''. Finally, the sentence which starts ``$\sigma dW_t$ ... " would read better if it started ``The term $\sigma dW_t$...".\\\\}}
			All the changes suggested above have been amended in the revised thesis on pg.~22 (highlighted in blue).
			
			\item{\textbf{Page 23, As it stands, you seem to be implying that an option can \emph{only} depend on the share price and time. You should make it clear that you are only really talking about European puts and calls at this stage.\\\\}}
			This is has been address on pg.~24 of the revised thesis with the phrase ``Assuming that the option depends only on the share price and time,...''.
			
			\item{\textbf{Page 23: The statement that $\phi \ : \ [0,\infty) \rightarrow [0,\infty)$ implies that an option can only have a non-negative payoff. I would argue that a modified European call, where you pay for the call at expiry but only if $S_T > K$, is an option even though it has a payoff which can be negative for some values of $S_T$.\\\\}}
			If we accept that the payoff function $\phi$ can be negative for some values of $S_T$, then all we would have to do is change $\phi$ to be $\phi \ : \ [0,\infty) \rightarrow \mathbb{R}$ and the analysis would remain identical.
			
			\item{\textbf{Page 23: It would also be a good idea to point out that $K > 0$ (otherwise you should include separate formulae for $K > 0$ and $K \leq 0 $ cases).\\\\}}
			We have added the assumption $K > 0$ on pg.~24 of the revised thesis.
			
			\item{\textbf{Page 23, final paragraph: I disagree with the assertion that ``To solve the system (2.2), (2.3), the most common approach is to implement a change of variables that transforms (2.2) into the standard one-dimensional heat equation...''. These days at least as many people would use the Feynman-Kac theorem and the solution of the underlying risk-neutral price SDE to find the solution of the problem.\\\\}}
			This statement has been altered to read ``one of the most common approaches is...'' on pg.~24 of the revised thesis.
			
			\item{\textbf{Page 24, after (2.6) and (2.7): This is the first time you have used $N$ and you should give the definition here.\\\\}}
			We have added ``(see Section 2.4)'' in parenthesis after the definition of (2.6) and (2.7) on pg.~25 of the revise thesis should the reader want a quick reference.
			
			\item{\textbf{Page 24: Please either give a referene to a proof of put-call parity or prove it yourself\\\\}}
			A proof has been referenced on right after (2.8) on pg.~25 of the revised thesis.
			
			\item{\textbf{Page 24: I don't understand what the sentence ``The structure we will present provides a clear interpretation of all the possible scenarios that can occur with American options.'' means or what it is you are trying to convey with it.\\\\}}
			This sentence has been reworded to ``The formulation we will present provides a clear framework for all the possible scenarios that can occur with American options'' on pg.~25 of the revised thesis.
			
			\item{\textbf{Page 24: In the second paragraph, you are talking about an American put, not a general American option. In this case you definitely need $r(t) > 0$ if you want $S^*(t)$ to exist (to the best of my knowledge, this is the first time you need this condition). You still need any conditions on the sign of $q(t)$, however.\\\\}}
			We have already stated in the preliminaries earlier that $r(t) > 0$ and $q(t)$ is nonnegative.
			
			\item{\textbf{Page 25, second paragraph: Change ``On the other hand, an American..." to ``An American...", as you have mentioned hands before.
			\\\\}}
			The phrase ``On the other hand'' has been removed.
			\item{\textbf{Page 25: At this point you are talking about an American call, so if you are going to insist that $r(t) > 0$, then you also need to explicitly state that $q(t) > 0$, otherwise $S^*(t)$ will not exist.
			\\\\}}
			We have added the statement ``We start off by assuming that $q(t) > 0$ (otherwise, the American call will be equal to the European call).'' as we commence the American call option analysis on pg.~26 of the revised thesis.
			
			\item{\textbf{Page 26, after (2.13)--(2.15): Jump diffusion is not introduced until after (2.17), so I don't see the point in saying ``... by Jamshidian [60] \emph{for a pure diffusion case.}''; we are still talking about diffusion only cases at this point.
			\\\\}}
			The sentence fragment ``for a pure diffusion case'' has been removed.
			
			\item{\textbf{Page 27, after (2.18) and (2.19): What is the $\mathbb{E}$ operator taking expectations with respect to ($W_t$, $N_t$ or $Y$)?\\\\}}
			This expectation operator is taking expectations with respect to $Y$. This has been added to pg.~28 of the revised thesis.
			\item{\textbf{Page 27, after (2.19): You might want to point out that Merton also perfectly hedged the diffusion risk but \emph{ignored} the market price of the jump risk in order to get (2.20).
			\\\\}}
			This suggestion has been added to pg.~29 of the revised thesis right before (2.20) as ``The extension to include jumps allowed for the diffusion risk to be perfectly hedged but ignored the market price of the jump risk.''.
			
			\item{\textbf{Page 28, \S2.2.1: Many people might refer to the functions defined by (2.24) and (2.25) as the discounted, risk-neutral, probability-density functions for the SDE (2.1).
			\\\\}}
			The terminology \emph{Black-Scholes kernel} was used because we were explicitly using results from~\cite{Rodrigo2006, Rodrigo2013}, where the authors use this term to denote (2.24) and (2.25).
			\item{\textbf{Page 29, \S2.3: If you do not define the Mellin transform and its inverse in \S1.5, you really \emph{must} define both the Mellin transform and \emph{its inverse} here. You should also correct the (incorrect) statement here that $\xi \in \mathbb{R}$ here so that it reads $\xi \in \mathbb{C}$.
			\\\\}}
			The Mellin transform and its inverse are now both defined in \S2.3 in the revised thesis. We have also changed $\xi$ to belong in $\mathbb{C}$ instead of $\mathbb{R}$.
			\item{\textbf{Page 29, second displayed equation: I think the second equation should read
				$$
					(x^2f'')(x) = x^2f''(x),
				$$
				the superscript that should be on the $x^2$ term on the left-hand side seems to have gone missing. There is an inherent problem with kind of notation; how do you define $(x^2f'')(y)$? Is it
				$$
					(x^2f'')(y) = x^2f''(y) \quad \text{or} \quad (x^2f'')(y) = y^2f''(y)?		
				$$
				Judging by (2.29) you would choose the latter.
			\\\\}}
			To clear the ambiguity, we have refined the functions at the bottom of pg.~30 of the revised the thesis by setting function $\text{id} = \text{id}(x)$. Thus $(\text{id}\cdot f')$ and $(\text{id}^2 \cdot f'')$ can be defined such that
			$$
				(\text{id}\cdot f')(x) = x f'(x), \quad (\text{id}^2\cdot f'')(x) = x^2 f''(x),
			$$
			which should eliminate any notation disputes.
			
			\item{\textbf{Page 30, \S 2.4:} Please insert the word \emph{standard} so that the first sentence reads ``Several properties of the \emph{standard} cumulative normal distribution $N$ ... ''.
			\\\\}
			The word \emph{standard} has been inserted.
			
						
			\item{\textbf{Page 31 et seq: } Here you are definitely assuming that $\mathscr{M}^{-1}$ exists and that $\mathscr{M}^{-1}(f)$ is unique. This is why you need to explicitly write down a formula for $\mathscr{M}^{-1}$ and state that its action (on some class of functions) produces a unique result.
			\\\\}
			The absence of the definition for the inverse Mellin transform has been addressed in previous comments.
			
			\item{\textbf{Page 36, \S3.1: } The line break for the title is not good. Please fix this if this is possible.
			\\\\}
			The line break has been fixed.
			
			\item{\textbf{Page 31, \S3.1: } I would change ``... the European option with an underlying \emph{subjected} to jump-diffusion dynamics.'' to ``... the European option with an underlying that is \emph{described by} jump-diffusion dynamics.''
			\\\\}
			The sentence now reads ``... the European option with an underlying that is \emph{described by} jump-diffusion dynamics.''
			\item{\textbf{Page 31, \S3.1: } Equation (3.1) would look better if the terminal condition appeared on a line by itself. \\\\}
			The terminal condition has been moved to a line by itself.
			
			\item{textbf{Page 37, first displayed equation} How do you know that $\mathbb{E}[Y^{-\xi}]$ exists?
			\\\\}
			This has been addressed in a previous comment whereby we added the assumption that $\mathbb{E}[Y^{-\xi}]$ is convergent.
			
			\item{\textbf{Page 38, \S3.2: } I would suggest that you state that all of the derivations in the section are formal (otherwise someone might ask you to prove them).
			\\\\}
			This has been appended to the beginning of \S3.2 of the revised thesis (pg.~40).
			
			\item{\textbf{Page 40, list of points: } In the first point the formula can only be applied if the distribution has a probability density function (otherwise (3.14) as it currently stands makes no sense).
			\\\\}
			It has been previously mentioned that $f$ is the PDF of some random variable $Y$, thus we have already satisfied the condition that the distribution must have a valid PDF (see pg.~29 of the revised thesis under \S2.2).
			
			\item{\textbf{Page 46, after the last displayed equation: } I would have written ``where $I$ is an interval with endpoints $a$ and $b > a$; the interval can be open, half-closed or closed''.
			\\\\}
			This has been reworded to Dr Dewynne's suggestion on pg.~48 of the revised thesis.
			
			\item{\textbf{Page 46, after the last displayed equation: } The statement that $I$ is the interval $[K,\infty)$, after the first displayed equation on Page 47, seems redundant.
			\\\\}
			This statement that $I$ is the interval $[K,\infty)$ has been removed.
			
			\item{\textbf{Page 50, \S3.6:} I'm not sure what it is you're trying to say in the statement ``However, the integrals (3.11) are all real since the Mellin transform inversion has been performed in a different manner to [90] where the inversion was completed via a complex integral''. Is this a simple statement or fact? Are you claiming that real integrals are easier to compute than complex ones (if so, why didn't you compute the integrals in (3.14) for the double exponential and gamma pdfs)? What you didn't do, however, is show that the $F_n$ term always exists.
			\\\\}
			For the statement in question, we wish to clarify that real integrals are in general easier to compute than complex integrals, but may not be necessarily easy to compute themselves. This is one of the key reasons why we decided to keep all our results in terms of real integrals versus complex integrals. The reason why we did not compute the integrals in (3.14) for the double exponential and gamma PDFs is because we merely wanted to demonstrate how our results would apply to these PDFs (which have results already found in literature). The purpose is more for comparison than anything else. The existence of $F_n$ has also been addressed in an earlier point of contention.
			
			\item{\textbf{Page 51:  Please give a reference to the ``recent empirical studies'' which suggest ``a better model for the asset process involves jumps following a double exponential distribution''.}
			\\\\}
			A reference has been provided and this sentence in question can now be found on pg.~52 of the revised thesis.
			
			\item{\textbf{Page 53, \S4.1.1: } Please give a definition of what it means for a function $\phi(x;x')$ to be homogeneous of degree one here (rather than half way down Page 54).
			\\\\}
			This is now in \S4.1.1 on pg.~55 of the revised thesis.
			
			\item{\textbf{Page 56, top of page: } I assume you mean $u(x',t;x) = v(x,t;x')$, otherwise what is the point of defining $u$?
			\\\\}
			This has been fixed on top of pg.~58 of the revised thesis.
			
			\item{\textbf{Page 73, (5.9): } It might be a good idea to use a different symbol to $\mathscr{J}$, which has already been used earlier.
			\\\\}
			We prefer to leave $\mathscr{J}$ in Eq.~(5.9) as it is for consistency of notation. In any case, we do not use the previous result Eq.~(3.13) here (i.e., different definitions).
			
			\item{\textbf{Page 74, \S5.1.2: } The ``lemma $n$'', ``Lemma $m$'' problem is particularly bad in this section (there is also an ``eq (3.18)'' on Page 75, whereas you usually write ``Eq.~(m.n)'').
			\\\\}
			The mixture of lowercase and uppercase for the word \emph{lemma} has now been fixed to be consistent along with fixing the outlier ``eq'' to be ``Eq''.
			
			\item{\textbf{Page 81, 2nd line from the bottom: } I would change ``Similar to how we handled $I_{3,a}$'' to ``Similarly to how we dealt with $I_{3,a}$''.
			\\\\}
			This has been changed on the bottom of pg.~83 of the revised thesis.
			
			\item{\textbf{Page 84, \S5.1.3:} I think this section could be shortened. It is not really that different to \S5.1.2 and it sort of gives the impression of ``cut and paste''.
			\\\\}
			\S5.1.3 was provided in full to complement \S5.1.2. This is because the formulas grew increasingly complicated that for the benefit of the reader, the details were provided in full so it would not be too overwhelming.
		\end{itemize}
\end{enumerate}

\begin{thebibliography}{100}
\bibitem{Chiarella2006} Chiarella, C. and Ziogas, A. (2006): ``A Fourier transform analysis of the American call op- tion on assets driven by jump-diffusion processes'', \emph{Quantitative Finance Research Centre, University of Technology Sydney, Research Paper No. 174 (2006)}.
\bibitem{Frontczak2013} Frontczak, R. (2013): ``Pricing options in jump-diffusion models using Mellin transforms'', \emph{Journal of Mathematical Finance}, Vol. 3, No. 3, pp.~336--373.
\bibitem{Kou2002} Kou, S. (2002): ``A jump-diffusion model for option pricing'', \emph{Management Science}, Vol. 48, No. 8, pp.~1086--1101.
	\bibitem{Kou2004} Kou, S. and H. Wang (2004): ``Option pricing under a double exponential jump-diffusion model'', \emph{Management Science}, Vol. 50, No. 9, pp. 1178--1192.
	\bibitem{Manaster1982} Manaster, S. and G. Koehler (1982): ``The calculation of implied variances from the Black-Scholes model: A Note'', \emph{The Journal of Finance}, Vol. 37, No. 1, pp. 227--230.
	\bibitem{Rodrigo2006} Rodrigo, M. R. and R. S. Mamon (2007): ``An application of Mellin transform techniques to a Black-Scholes equation problem'', \emph{Analysis and Applications}, Vol. 5, No. 1, pp. 51--66.
	\bibitem{Rodrigo2013} Rodrigo, M. R. (2014): ``Approximate ordinary differential equations for the optimal exercise boundaries of American put and call options'', \emph{European Journal of Applied Mathematics}, Vol. 25, No. 1, pp. 27--43.
	\bibitem{Wilmott1993} Wilmott, P. and Dewynne, J. and Howison, S. (1995): ``The Mathematics of Financial Derivatives: A Student's Introduction'', \emph{Cambridge University Press}.

\end{thebibliography}
\end{document}